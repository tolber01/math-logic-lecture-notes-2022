\subsection{Полнота и непротиворечивость теорий первого порядка}
\begin{lemma}
    Во всякой теории первого порядка любая теорема логически общезначима.
\end{lemma}
Полнота теории первого порядка должна означать доказуемость любой логически общезначимой формулы. Но логическая общезначимость определяется истинностью на интерпретациях. Поэтому установление полноты требует применения средств теории моделей.
\begin{lemma}
    Всякая логически общезначимая формула теории первого порядка $T$ является теоремой теории $T$.
\end{lemma}
\begin{theorem}[\textit{теорема Гёделя о полноте}]
    Во всяком ИП первого порядка теоремами являются те и только те формулы, которые логически общезначимые.
\end{theorem}
\begin{theorem}[\textit{непротиворечивость ИП}]
    Всякое ИП первого порядка непротиворечиво.
\end{theorem}
\begin{proof}
    Поставим в соответствие каждой формуле $A$ теории ИП первого порядка формулу $h(A)$, в которой опущены все кванторы и термы с соответствующими скобками и запятыми. В таком случае $h(A)$ станет пропозициональной формулой ИВ. Пример: $\forall xA\big(f(x), y\big) \supset \exists z\neg C(z) \overset{h}{\longmapsto} A \supset \neg C$. Фактически $h(A)$~---~интерпретация $A$ на однооэлементной области интерпретации. Ясно, что верны следующие утверждения про $h$:
    \begin{enumerate}
        \item $h(\neg A)$ совпадает с $\neg h(A)$.
        \item $h(A \supset B)$ совпадает с $h(A) \supset h(B)$ (то же самое верно для остальных связок).
        \item Если $A$~---~аксиома A1--A5, то $h(A)$~---~тавтология.
        \item Если $h(A)$ и $h(A \supset B)$~---~тавтологии, то и $h(B)$~---~тавтология.
        \item Если $h(A)$~---~тавтология, то и $h(\forall x_iA)$~---~тавтология, какова бы ни была переменная $x_i$.
    \end{enumerate}
    Теорему теперь докажем от противного. Предположим, что существует утверждение $A$, для которого имеют место обе выводимости в теории ИП: $\vdash A$ и $\vdash \neg A$. Из перечисленных выше замечаний про $h$ заключаем, что $h(B)$ для любой теоремы $B$ суть тавтология. То есть $h(A)$ и $h(\neg A)$ являются тавтологиями, а значит $h(A)$ и $\neg h(A)$ тоже являются тавтологиями, что невозможно в силу непротиворечивости ИВ. Итак, для любой теории ИП первого порядка не существует утверждения $A$, для которого справедливы обе выводимости $\vdash A$ и $\vdash \neg A$, значит всякая теория ИП первого порядка непротиворечива.
\end{proof}