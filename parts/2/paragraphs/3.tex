\subsection{Теории первого порядка}
Для построения теории используем введённый в \ref{par:first_order_logic} алфавит. При этом положим, что множества предикатных букв и предметных переменных не пусты, множества функциональных символов и предметных констант могут быть пустыми.

Термы, элементарные формулы и формулы определяются так же.

\begin{definition*}
    Терм $t$ называется \textit{свободным для переменной $x_i$ в формуле $A$}, если никакое свободное вхождение $x_i$ в $A$ не лежит в области действия никакого квантора $\forall x_j$, где $x_j$~---~переменная, входящая в терм $t$.
\end{definition*}
Иными словами, терм $t$ свободен для переменной $x_i$ в формуле $A$, если при подстановке $t$ в $A$ вместо всех свободных вхождений $x_i$ никакая переменная в подставленном терме $t$ не свяжется никаким квантором в формуле $A$. Ясно, что если в $A$ нет свободных вхождений переменной $x_i$, то любой терм будет свободным относительно этой переменной. Также терм $t = x_i$, как нетрудно сообразить, будет свободным относительно $x_i$, какова бы ни была формула $A$.

Аксиомами теории будут следующие формулы:
\begin{enumerate}[label=(А\arabic*)]
    \item $A \supset (B \supset A)$;
    \item $\big(A \supset (B \supset C)\big) \supset \big((A \supset B) \supset (A \supset C)\big)$;
    \item $(\neg B \supset \neg A) \supset \big((\neg B \supset A) \supset B\big)$;
    \item $\forall x_iA(x_i) \supset A(t)$, где $t$~---~терм, свободный для $x_i$ в формуле $A(x_i)$, а $A(t)$~---~формула, в которой все вхождения переменной $x_i$ заменены на $t$;
    \item $\forall x_i(A \supset B) \supset (A \supset \forall x_iB)$, где $A$ не содержит свободные вхождения переменной $x_i$.
\end{enumerate}
Различные конкретные теории исчисления предикатов могут содержать ещё аксиомы в дополнение к приведённым выше (\textit{собственные} аксиомы). Теория первого порядка без собственных аксиом называется \textit{исчислением предикатов первого порядка} (\textit{ИП}). \\
Первый порядок в названии выражается в том, что в предикаты можно подставлять исключительно термы, определённые в \ref{par:first_order_logic}, а в кванторах могут записываться только индивидные переменные. В теориях более высокого порядка допускается взятие кванторов по предикатным символам, а так же подстановка в предикаты других предикатов.

Правила вывода:
\begin{enumerate}[label=\arabic*)]
    \item MP
    \[
        \frac{A,\ A \supset B}{B};
    \]
    \item обобщение (Gen)
    \[
        \frac{A}{\forall x_iA}.
    \]
\end{enumerate}

\begin{lemma}\label{th:tautology_is_theorem}
    Если формула $A$ произвольной теории первого порядка $T$ есть частный случай тавтологии, то $A$ есть теорема $T$ и может быть выведена с применением одних только схем аксиом А1, A2, А3 и правила MP.
\end{lemma}