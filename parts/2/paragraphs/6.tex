\subsection{Некоторые дополнительные правила вывода в ИП}
Для упрощения работы с конкретными теориями первого порядка полезно доказать некоторые дополнительные факты о них. Пусть далее $T$~---~произвольная теория первого порядка.

\subsubsection{Правила индивидуализации и существования}
\textbf{Правило индивидуализации}: если терм $t$ свободен для $x$ в формуле $A(x)$, то $\forall xA(x) \vdash A(t)$.
\begin{proof}\leavevmode
    \begin{enumerate}
        \item $\forall xA(x) \supset A(t)$ (А4, её использование правомерно, так как $t$ свободен для $x$ в $A(x)$);
        \item $\forall xA(x)$ (гипотеза);
        \item $A(t)$ (из 2, 1 по MP).
    \end{enumerate}
\end{proof}

\begin{lemma}\label{th:exists_theorems}
    Если $A$ и $B$~---~формулы и предметная переменная $x$ не свободна в $A$, то:
    \begin{enumerate}[label=\arabic*)]
        \item $\vdash A \supset \forall xA$ (следовательно, по аксиоме А4 получается $\vdash A \equiv \forall xA$);
        \item $\vdash \exists xA \supset A$ (следовательно, по нижеследующему правилу существования $\vdash \exists xA \equiv A$);
        \item $\vdash \forall x(A \supset B) \equiv (A \supset \forall xB)$;
        \item $\vdash \forall x(B \supset A) \equiv (\exists xB \supset A)$.
    \end{enumerate}
\end{lemma}
\begin{proof}\leavevmode
    \begin{enumerate}[label=\arabic*)]
        \item Ясно, что по правилу Gen справедлива выводимость $A \vdash \forall xA$. Так как в данном случае переменная $x$ не свободна в $A$ и она не свяжется квантором $\forall x$, правомерно использование теоремы дедукции, дающее искомое утверждение: $\vdash A \supset \forall xA$.

        \item Формула $\exists xA \supset A$ расшифровывается как $\neg\forall x(\neg A) \supset A$. Докажем вспомогательную выводимость $\neg\forall x(\neg A) \vdash A$:
        \begin{enumerate}[label=\arabic*.]
            \item $\big(\neg A \supset \neg\forall x(\neg A)\big) \supset \Big(\big(\neg A \supset \forall x(\neg A)\big) \supset A\Big)$ (А3);
            \item $\neg\forall x(\neg A) \supset \big(\neg A \supset \neg\forall x(\neg A)\big)$ (А1);
            \item $\neg\forall x(\neg A)$ (гипотеза);
            \item $\neg A \supset \neg\forall x(\neg A)$ (из 3, 2 по MP);
            \item $\big(\neg A \supset \forall x(\neg A)\big) \supset A$ (из 5, 1 по MP);
            \item $\neg A \supset \forall x(\neg A)$ (доказанный п.1: ясно, что если $x$ не свободна в $A$, то $x$ не свободна и в $\neg A$);
            \item $A$ (из 6, 5 по MP).
        \end{enumerate}
        Ясно, что поскольку $x$ не является свободной переменной в $A$, то в данной ситуации также можно применить теорему дедукции и получить требуемое: $\neg\forall x(\neg A) \supset A$, то есть $\exists xA \supset A$.

        \item $\forall x(A \supset B) \supset (A \supset \forall xB)$~---~аксиома А5, поскольку $x$ не свободна в $A$. Докажем вспомогательную выводимость $A \supset \forall xB, A \vdash B$: 
        \begin{enumerate}[label=\arabic*.]
            \item $A \supset \forall xB$ (гипотеза);
            \item $A$ (гипотеза);
            \item $\forall xB$ (из 2, 1 по MP);
            \item $\forall xB \supset B$ (А4, её использование законно, так как терм $t = x$ свободен для $x$ в любой формуле);
            \item $B$ (из 3, 4 по MP).
        \end{enumerate}
        Так как в выводе выше ни разу не было использовано правило Gen, можно применить теорему дедукции и получить $A \supset \forall xB \vdash A \supset B$. По правилу Gen получаем, что $A \supset B \vdash \forall x(A \supset B)$, а значит по правилу сечения справедливо $A \supset \forall xB \vdash \forall x(A \supset B)$. \\
        Ясно, исходя из предыдущих выводов и доказательства теоремы дедукции, что существует вывод $A \supset \forall xB \vdash \forall x(A \supset B)$, в котором правило Gen применяется только к переменной $x$, которая, во-первых, не свободна в $A$, во-вторых, не свободна в $\forall xB$, откуда $x$ не свободна и в $A \supset \forall xB$, а значит мы снова имеем право применить теорему дедукции и получить желанное $\vdash (A \supset \forall xB) \supset \forall x(A \supset B)$. \\
        Наконец, используя правило конъюнкции, доказываемое чуть дальше, и правило сечения, получаем, что
        \[
            \vdash \big(\forall x(A \supset B) \supset (A \supset \forall xB)\big) \land \big((A \supset \forall xB) \supset \forall x(A \supset B)\big),
        \]
        что равносильно искомому $\vdash \forall x(A \supset B) \equiv (A \supset \forall xB)$.

        \item \texttt{ToDo...}
    \end{enumerate}
\end{proof}

\textbf{Правило существования} (контрапозиция правила индивидуализации): если терм $t$ свободен для $x$ в $A(x)$, то $\vdash A(t) \supset \exists xA(x)$ и, следовательно, $A(t) \vdash \exists xA(x)$.
\begin{proof}
    Формула $A(t) \supset \exists xA(x)$ является сокращением записи $A(t) \supset \neg \forall x\big(\neg A(x)\big)$. Предъявим вывод этой формулы:
    \begin{enumerate}
        \item $\forall x\big(\neg A(x)\big) \supset \neg A(t)$ (А4, её использование правомерно, так как терм $t$ свободен для $x$ в $A(x)$, а значит и в $\neg A(x)$);
        \item $(\mathcal{A} \supset \neg \mathcal{B}) \supset (\mathcal{B} \supset \neg \mathcal{A})$ (теорема, являющаяся частным случаем тавтологии по лемме \ref{th:tautology_is_theorem});
        \item $A(t) \supset \neg\forall x\big(\neg A(x)\big)$ (из 1, 2 по MP, полагая $\mathcal{A} \leftrightharpoons \forall x\big(\neg A(x)\big)$ и $\mathcal{B} \leftrightharpoons A(t)$).
    \end{enumerate}
    Так как доказано $A(t) \supset \exists xA(x)$, легко доказать и $A(t) \vdash \exists xA(x)$:
    \begin{enumerate}
        \item $A(t)$ (гипотеза);
        \item $A(t) \supset \exists xA(x)$ (выведенная теорема);
        \item $\exists xA(x)$ (из 1, 2 по MP).
    \end{enumerate}
\end{proof}

\textbf{Правило конъюнкции}: $A, B \vdash A \land B$.
\begin{proof}Вспомним, что $A \land B$~---~сокращение записи $\neg(A \supset \neg B)$. Тем самым, требуется доказать выводимость $A, B \vdash \neg(A \supset \neg B)$. Доказательство имеет вид:
    \begin{enumerate}
        \item $A$ (гипотеза);
        \item $B$ (гипотеза);
        \item $A \supset \big(B \supset \neg(A \supset \neg B)\big)$ (тавтология, являющаяся теоремой по лемме \ref{th:tautology_is_theorem});
        \item $B \supset \neg(A \supset \neg B)$ (из 1, 3 по MP);
        \item $\neg(A \supset \neg B)$ (из 2, 4 по MP).
    \end{enumerate}
\end{proof}

\begin{lemma}\label{th:equivalence_distribution}
    Для любых формул $A$ и $B$ справедливо $\vdash \forall x(A \equiv B) \supset (\forall xA \equiv \forall xB)$.
\end{lemma}
\begin{proof}
    Вывод $\forall x(A \equiv B), \forall xA \vdash \forall xB$: 
    \begin{enumerate}
        \item $\forall x(A \equiv B)$ (гипотеза);
        \item $\forall xA$ (гипотеза);
        \item $A \equiv B$ (из 1 по правилу индивидуализации, поскольку терм $t = x$ свободен относительно $x$ в любой формуле);
        \item $A$ (из 2 по правилу индивидуализации);
        \item $(A \equiv B) \supset (A \supset B)$ (тавтология, являющаяся теоремой согласно лемме \ref{th:tautology_is_theorem});
        \item $A \supset B$ (из 3, 5 по MP);
        \item $B$ (из 4, 6 по MP);
        \item $\forall xB$ (из 7 по Gen).
    \end{enumerate}
    Далее по теореме дедукции $\forall x(A \equiv B) \vdash \forall xA \supset \forall xB$. Аналогично доказывается $\forall x(A \equiv B) \vdash \forall xB \supset \forall xA$. По правилу конъюнкции и правилу сечения получаем $\forall x(A \equiv B) \vdash \forall xA \equiv \forall xB$. Теперь по теореме дедукции приходим к доказываемому утверждению.
\end{proof}

\subsubsection{Теорема эквивалентности}
\begin{theorem}[\textit{теорема эквивалентности}]
    Если $B$ есть подформула $A$ и $A'$ есть результат замены каких-либо (может быть, ни одного) вхождений $B$ формулой $C$ и если всякая свободная переменная формулы $B$ или $C$, являющаяся одновременно связанной переменной формулы $A$, встречается в списке $y_1, y_2, \dots, y_k$, то
    \[
        \vdash \forall y_1\forall y_2\dots\forall y_k(B \equiv C) \supset (A \equiv A').
    \]
\end{theorem}
\begin{proof}
    Рассмотрим тривиальные случаи:
    \begin{itemize}
        \item Пусть ни одно вхождение $B$ не заменяется на $C$. Тогда $A'$ просто совпадает с $A$. Требуется в таком случае доказать, что
        \[
            \vdash \forall y_1\forall y_2\dots\forall y_k(B \equiv C) \supset (A \equiv A).
        \]
        Упомянутая теорема имеет вид $D \supset (A \equiv A)$, где $D \leftrightharpoons \forall y_1\forall y_2\dots\forall y_k(B \equiv C)$. Ясно, что это тавтология, откуда по лемме \ref{th:tautology_is_theorem} она является теоремой ИП.
        
        \item Пусть теперь $B$ совпадает с $A$ и это единственное вхождение заменяется на $C$. Тогда надо доказать
        \[
            \vdash \forall y_1\forall y_2\dots\forall y_k(B \equiv C) \supset (B \equiv C).
        \]
        Это следует из более общего утверждения: $\vdash \forall y_1\forall y_2\dots\forall y_k\mathcal{A} \supset \mathcal{A}$, следующего, в свою очередь, по индукции из правила индивидуализации. Взяв вместо $\mathcal{A}$ формулу $B \equiv C$, получаем требуемое.
    \end{itemize}

    Далее считаем, что хотя бы одно вхождение $B$ заменяется, и $B$ не совпадает с $A$ (то есть $B$~---~собственная подформула $A$).
    
    Доказательство проведём индукцией по числу связок и кванторов в формуле $A$.

    \textbf{База}: в $A$ одна связка или один квантор. Имеются варианты:
    \begin{itemize}
        \item \underline{В $A$ всего один квантор всеобщности}. Иными словами, $A$ имеет вид $\forall xP$, где $P$~---~элементарная формула. \\
        Единственная собственная подформула $A$, как легко видеть, есть $P$, а значит $P$ совпадает с $B$. Следовательно, $A'$ имеет вид $\forall xC$. Тогда нужно доказать $\vdash \forall y_1\forall y_2\dots\forall y_k(B \equiv C) \supset (\forall xB \equiv \forall xC)$. \\
        Ясно, что поскольку $x$ является связанной переменной в $A$, то она входит в список $y_1, y_2, \dots, y_k$. Пусть $x$ есть $y_i$. В таком случае от нас требуется доказать, что
        \begin{equation}\label{eq:base_1}
            \vdash \forall y_1\forall y_2\dots\forall y_k(B \equiv C) \supset (\forall y_iB \equiv \forall y_iC).
        \end{equation}
        Многократно используя доказанное в \ref{par:deduction_th} утверждение $\forall x\forall yA(x, y) \vdash \forall y\forall xA(x, y)$, мы можем получить \eqref{eq:base_1} из выводимости
        \begin{equation}\label{eq:base_2}
            \vdash \forall y_1\forall y_2\dots\forall y_{i - 1}\forall y_{i + 1}\dots\forall y_k\forall y_i(B \equiv C) \supset (\forall y_iB \equiv \forall y_iC). 
        \end{equation}
        Согласно лемме \ref{th:equivalence_distribution}, справедливо $\vdash \forall y_i(B \equiv C) \supset (\forall y_iB \equiv \forall y_iC)$, а значит имеет место 
        \begin{equation}\label{eq:base_3}
            \forall y_i(B \equiv C) \vdash \forall y_iB \equiv \forall y_iC.
        \end{equation} 
        Согласно уже упомянутому обобщённому правилу индивидуализации, имеем 
        \begin{equation}\label{eq:base_4}
            \forall y_1\forall y_2\dots\forall y_{i - 1}\forall y_{i + 1}\dots\forall y_k\forall y_i(B \equiv C) \vdash \forall y_i(B \equiv C). 
        \end{equation}
        Тогда по правилу сечения из \eqref{eq:base_3} и \eqref{eq:base_4} получаем $\forall y_1\forall y_2\dots\forall y_{i - 1}\forall y_{i + 1}\dots\forall y_k\forall y_i(B \equiv C) \vdash \forall y_iB \equiv \forall y_iC$. Из полученной выводимости по теореме дедукции получаем \eqref{eq:base_2}, а из \eqref{eq:base_2}, как было уже замечено, получаем искомое \eqref{eq:base_1}. Законность применения теоремы дедукции здесь обусловлена тем, при выводе правило Gen применялось только к переменным $y_1, y_2, \dots, y_k$, которые, с одной стороны, не свободны в $A$, и, с другой стороны, не свободны в $\forall y_1\forall y_2\dots\forall y_{i - 1}\forall y_{i + 1}\dots\forall y_k\forall y_i(B \equiv C)$.

        \item \underline{В $A$ есть одна связка и только она}, то есть $A$ имеет вид $\neg P$ или $P \supset Q$, где $P, Q$~---~элементарные формулы. Рассмотрим два этих подслучая отдельно:
        \begin{itemize}
            \item \underline{$A$ имеет вид $\neg P$}, тогда $P$ совпадает с $B$, а $A'$ есть $\neg C$. Фактически требуется доказать, что
            \begin{equation}\label{eq:base_5}
                \vdash \forall y_1\forall y_2\dots\forall y_k(B \equiv C) \supset (\neg B \equiv \neg C).
            \end{equation}
            Применяя обобщённое правило индивидуализации, получаем, что 
            \begin{equation}\label{eq:base_6}
                \forall y_1\forall y_2\dots\forall y_k(B \equiv C) \vdash B \equiv C. 
            \end{equation}
            Записав далее очевидную тавтологию $(B \equiv C) \supset (\neg B \equiv \neg C)$, из которой вытекает $B \equiv C \vdash \neg B \equiv \neg C$, по правилу сечения из \eqref{eq:base_6} и последовавшей из написанной тавтологии выводимости получаем, что
            \begin{equation}\label{eq:base_7}
                \forall y_1\forall y_2\dots\forall y_k(B \equiv C) \vdash \neg B \equiv \neg C.
            \end{equation}
            Желанное \eqref{eq:base_5} вытекает из \eqref{eq:base_7} по теореме дедукции (правомерность её использования такая же, как в случае, когда $A$ имеет вид $\forall xP$).

            \item \underline{$A$ имеет вид $P \supset Q$}, где либо $P$ совпадает с $B$, либо $Q$ совпадает с $B$, либо выполнено и то, и другое. Доказательство для этого подслучая такое же, как для предудщего подслучая, только в качестве тавтологий нужно рассмотреть соответственно $(B \equiv C) \supset \big((B \supset Q) \equiv (C \supset Q)\big)$, $(B \equiv C) \supset \big((P \supset B) \equiv (P \supset C)\big)$ и $(B \equiv C) \supset \big((B \supset B) \equiv (C \supset C)\big)$.
        \end{itemize}
    \end{itemize}

    Тем самым, проверена база индукции.

    \textbf{Предположение индукции}: пусть теорема верна для любой формулы с меньшим числом связок и кванторов, чем в $A$. 

    \textbf{Шаг индукции}: докажем утверждение для $A$. Ясно, что $A$ не может быть атомарной формулой, поскольку тогда $A$ не имела бы собственных подформул, среди которых по нашему предположению есть $B$. В таком случае имеются следующие варианты:
    \begin{itemize}
        \item \underline{$A$ есть $\neg P$}. Тогда $A'$ есть $\neg P'$, где $P'$~---~результат подстановки в $P$ некоторого числа (как минимум одной) формул $C$ вместо вхождений подформулы $B$. Ясно, что в $P$ содержится меньше связок и кванторов, чем в $A$, тогда для $P$ применимо индукционное предположение:
        \begin{equation}\label{eq:step_1}
            \vdash \forall y_1\forall y_2\dots\forall y_k(B \equiv C) \supset (P \equiv P') \Longrightarrow \forall y_1\forall y_2\dots\forall y_k(B \equiv C) \vdash P \equiv P'.
        \end{equation}
        Имея в виду тавтологию
        \[
            (\mathcal{A} \equiv \mathcal{B}) \supset (\neg \mathcal{A} \equiv \neg \mathcal{B}),
        \]
        являющуюся теоремой согласно лемме \ref{th:tautology_is_theorem}, можем получить следующее:
        \begin{equation}\label{eq:step_2}
            \vdash (P \equiv P') \supset (\neg P \equiv \neg P') \Longrightarrow\ P \equiv P' \vdash \neg P \equiv \neg P'.
        \end{equation}
        Из \eqref{eq:step_1} и \eqref{eq:step_2} по правилу сечения можно заключить, что
        \[
            \forall y_1\forall y_2\dots\forall y_k(B \equiv C) \vdash \neg P \equiv \neg P' \Longleftrightarrow \forall y_1\forall y_2\dots\forall y_k(B \equiv C) \vdash A \equiv A'.
        \]
        Искомая выводимость получается из предыдущей примененим теоремы дедукции:
        \[
            \vdash \forall y_1\forall y_2\dots\forall y_k(B \equiv C) \supset (A \equiv A').
        \]
        Правомерность применения теоремы дедукции обосновывается тем, что, согласно нашему доказательству, правило Gen использовалось только к переменным из списка $y_1, y_2, \dots, y_k$ (свободные переменные $B \equiv C$), а они, как видно в выводимости выше, уже связаны кванторами.

        \item \underline{$A$ есть $P \supset Q$}. Тогда $A'$ имеет вид $P' \supset Q'$, где $P', Q'$~---~результаты замены некоторых (как минимум одного) вхождений $B$ на $C$ в $P$ и $Q$ соответственно. Ясно, что в $P$ и $Q$ меньше связок и кванторов, чем в $A$. Тогда для этих формул применимо предположение индукции:
        \begin{equation}\label{eq:step_3}
            \begin{cases}
                \vdash \forall y_1\forall y_2\dots\forall y_k(B \equiv C) \supset (P \equiv P') \Longrightarrow \forall y_1\forall y_2\dots\forall y_k(B \equiv C) \vdash P \equiv P', \\
                \vdash \forall y_1\forall y_2\dots\forall y_k(B \equiv C) \supset (Q \equiv Q') \Longrightarrow \forall y_1\forall y_2\dots\forall y_k(B \equiv C) \vdash Q \equiv Q'.
            \end{cases}
        \end{equation}
        Запишем следующую тавтологию:
        \[
            (\mathcal{A} \equiv \mathcal{B}) \land (\mathcal{C} \equiv \mathcal{D}) \supset \big((\mathcal{A} \supset \mathcal{C}) \equiv (\mathcal{B} \supset \mathcal{D})\big).
        \]
        Пользуясь этой тавтологией и леммой \ref{th:tautology_is_theorem}, получаем следующее:
        \begin{equation}\label{eq:step_4}
            \vdash (P \equiv P') \land (Q \equiv Q') \supset \big((P \supset Q) \equiv (P' \supset Q')\big) \Longrightarrow (P \equiv P') \land (Q \equiv Q') \vdash (P \supset Q) \equiv (P' \supset Q').
        \end{equation}
        Теперь по правилу конъюнкции из \eqref{eq:step_3} получаем выводимость:
        \begin{equation}\label{eq:step_5}
            \forall y_1\forall y_2\dots\forall y_k(B \equiv C) \vdash (P \equiv P') \land (Q \equiv Q').
        \end{equation}
        Далее из \eqref{eq:step_5} и \eqref{eq:step_4} и правила сечения заключаем
        \[
            \forall y_1\forall y_2\dots\forall y_k(B \equiv C) \vdash (P \supset Q) \equiv (P' \supset Q') \Longleftrightarrow \forall y_1\forall y_2\dots\forall y_k(B \equiv C) \vdash A \equiv A'.
        \]
        Теорема дедукции, применённая к полученной выше выводимости, даёт нужное утверждение. Правомерность применения теоремы дедукции обосновывается так же, как в предыдущем пункте.

        \item \underline{$A$ имеет вид $\forall x_iP$}. Тогда $A'$ есть $\forall x_iP'$, где $P'$~---~результат замены некоторых вхождений $B$ на $C$ в формуле $P$. Ясно, что в $P$ на один квантор меньше, чем в $A$, значит для $P$ справедливо индукционное предположение:
        \begin{equation}\label{eq:step_6}
            \vdash \forall y_1\forall y_2\dots\forall y_k(B \equiv C) \supset (P \equiv P') \Longrightarrow \forall y_1\forall y_2\dots\forall y_k(B \equiv C) \vdash P \equiv P'.
        \end{equation}
        Требуется показать, что
        \[
            \vdash \forall y_1\forall y_2\dots\forall y_k(B \equiv C) \supset (\forall x_iP \equiv \forall x_iP').
        \]
        Отметим, что переменная $x_i$ не встречается свободно в $\forall y_1\forall y_2\dots\forall y_k(B \equiv C)$. Если бы она входила свободно в эту формулу, то она была бы свободной переменной $B$ или $C$, но $x_i$ одновременно с этим является связанной переменной в формуле $A$, а значит $x_i$ входит в список $y_1, y_2, \dots, y_k$, согласно условию теоремы. Это значит, что она была бы связана в $\forall y_1\forall y_2\dots\forall y_k(B \equiv C)$. \\
        Далее по правилу Gen $P \equiv P' \vdash \forall x_i(P \equiv P')$, что из леммы \ref{th:equivalence_distribution}, правила сечения и (\ref{eq:step_6}) даёт
        \[
            \forall y_1\forall y_2\dots\forall y_k(B \equiv C) \vdash \forall x_iP \equiv \forall x_iP' \Longleftrightarrow \forall y_1\forall y_2\dots\forall y_k(B \equiv C) \vdash A \equiv A'.
        \]
        Поскольку правило обобщения здесь применялось к $x_i$, которая не является свободной в $\forall y_1\forall y_2\dots\forall y_k(B \equiv C)$, то можно применить теорему дедукции и заключить желаемое:
        \[
            \vdash \forall y_1\forall y_2\dots\forall y_k(B \equiv C) \supset (A \equiv A').
        \]
    \end{itemize}
    Итак, все случаи разобраны, а значит индукционный шаг доказан. Тем самым, индукцией по числу связок и кванторов в формуле $A$ доказана требуемая теорема.
\end{proof}
\begin{corollary}[\textit{теорема о замене}]
    Пусть формулы $A, B, C$ и $A'$ удовлетворяют условию теоремы эквивалентности. Тогда:
    \begin{enumerate}
        \item Если $\vdash B \equiv C$, то $\vdash A \equiv A'$.
        \item Если $\vdash B \equiv C$ и $\vdash A$, то $\vdash A'$.
    \end{enumerate}
\end{corollary}
\begin{definition*}
    Формулы $A(x_i)$ и $A(x_j)$ (переменные $x_i$ и $x_j$ не совпадают, $A(x_j)$ получается из $A(x_i)$ подстановкой $x_j$ вместо всех свободных вхождений $x_i$) называются \textit{подобными}, если $x_j$ свободна для $x_i$ в $A(x_i)$ и $A(x_i)$ не имеет свободных вхождений $x_j$.
\end{definition*}

Если $A(x_i)$ и $A(x_j)$ подобны, то $x_i$ свободна для $x_j$ в $A(x_j)$, и $A(x_j)$ не имеет свободных вхождений $x_i$. Таким образом, подобие формул симметрично. Следовательно, $A(x_i)$ и $A(x_j)$ подобны тогда и только тогда, когда $A(x_j)$ имеет свободные вхождения $x_j$ в точности в тех местах, в которых $A(x_i)$ имеет свободные вхождения $x_i$.
\begin{lemma}\label{th:var_rename}
    Если формулы $A(x_i)$ и $A(x_j)$ подобны, то $\vdash \forall x_iA(x_i) \equiv \forall x_jA(x_j)$.
\end{lemma}
\begin{corollary}[\textit{переименование связанных переменных}]
    Если $\forall xB(x)$ есть подформула формулы $A$, формула $B(y)$ подобна $B(x)$ и $A'$ есть результат замены по крайней мере одного вхождения $\forall xB(x)$ в $A$ на $\forall yB(y)$, то $\vdash A \equiv A'$.
\end{corollary}

\subsubsection{Правило C}
Пусть надо доказать $\exists x\big(B(x) \supset C(x)\big), \forall xB(x) \vdash \exists xC(x)$. \\
\textbf{Интуитивное доказательство}:
\begin{enumerate}
    \item $\exists x\big(B(x) \supset C(x)\big)$ (гипотеза);
    \item $\forall xB(x)$ (гипотеза);
    \item $\color{red}B(a) \supset C(a)$ (из 1, выбор некоторого $a$);
    \item $B(a)$ (из 2, по правилу индивидуализации, терм $t = a$ свободен для любой переменной в любой формуле);
    \item $C(a)$ (из 4, 3 по MP);
    \item $\exists xC(x)$ (из 5 по правилу существования).
\end{enumerate}
Можно то же самое доказать \textbf{без произвольного выбора} элемента на шаге 3. \\
Построим вывод $\forall xB(x), \forall x\neg C(x) \vdash \forall x\neg\big(B(x) \supset C(x)\big)$:
\begin{enumerate}
    \item $\forall xB(x)$ (гипотеза);
    \item $\forall x\neg C(x)$ (гипотезы);
    \item $B(x)$ (из 1 по правилу индивидуализации);
    \item $\neg C(x)$ (из 2 по правилу индивидуализации);
    \item $B(x) \land \neg C(x)$ (из 3, 4 по правилу конъюнкции);
    \item $B(x) \land \neg C(x) \supset \neg\big(B(x) \supset C(x)\big)$ (частный случай тавтологии $\mathcal{A} \land \neg \mathcal{B} \supset \neg(\mathcal{A} \supset \mathcal{B})$);
    \item $\neg\big(B(x) \supset C(x)\big)$ (из 5, 6 по MP);
    \item $\forall x\neg\big(B(x) \supset C(x)\big)$ (из 7 по Gen).
\end{enumerate}
По теореме дедукции получаем
\[
    \forall xB(x) \vdash \forall x\neg C(x) \supset \forall x\neg\big(B(x) \supset C(x)\big).
\]
Отсюда с помощью тавтологии $(\mathcal{A} \supset \mathcal{B}) \supset (\neg \mathcal{B} \supset \neg \mathcal{A})$ по правилу MP имеем
\[
    \forall xB(x) \vdash \exists x\big(B(x) \supset C(x)\big) \supset \exists xC(x).
\]
Отсюда можно получить 
\[
    \exists x\big(B(x) \supset C(x)\big), \forall xB(x) \vdash \exists xC(x).
\]
Правило C (choice) позволяет переходить от $\exists xA(x)$ к $A(a)$.

\textbf{Вывод с применением правила C}. \\
$\Gamma \vdash_C A$ тогда и только тогда, когда существует вывод $B_1, B_2, \dots, B_n$ формулы $A$, в котором:
\begin{enumerate}
    \item Для любого $i$ формула $B_i$ есть либо аксиома, либо гипотеза из $\Gamma$, либо непосредственное следствие по MP или Gen из каких-либо предшествующих формул, либо $C(b)$, где $b$~---~новая предметная константа, и формуле $B_i$ предшествует формула $\exists xC(x)$ (правило C).
    \item В качестве аксиом в 1 допускаются также всевозможные логические аксиомы с новыми предметными константами, уже введёнными ранее по правилу C.
    \item Не допускается применение правила Gen по переменным, свободным хотя бы в одной формуле вида $\exists xC(x)$, к которой ранее было применено правило C.
    \item Формула $A$ не содержит новых предметных констант, введённых с помощью правила C.
\end{enumerate}
Пример вывода с применением правила C без п.3:
\[
    \forall x\exists yA_1^2(x, y) \vdash_C \exists y\forall xA_1^2(x, y):
\]
\begin{enumerate}
    \item $\forall x\exists yA_1^2(x, y)$ (гипотеза);
    \item $\exists yA_1^2(x, y)$ (из 1 по индивидуализации);
    \item $A_1^2(x, b)$ (из 2 по правилу C);
    \item $\forall xA_1^2(x, b)$ (из 3 по правилу Gen);
    \item $\exists y\forall xA_1^2(x, y)$ (из 4 по правилу существования).
\end{enumerate}
Между тем, данная выводимость не является верной: достаточно рассмотреть интерпретацию с областью $D = \mathbb{R}$ и $A_1^2(x, y) \Longleftrightarrow x < y$.
\begin{theorem}\label{th:c_rule_inference}
    Если $\Gamma \vdash_C A$, то $\Gamma \vdash A$.
\end{theorem}
\begin{proof}
    Пусть $\exists y_1C_1(y_1), \dots, \exists y_kC_k(y_k)$~---~формулы (в порядке появления) в выводе $\Gamma \vdash_C A$, к которым применено правило C, а $a_1, a_2, \dots, a_k$~---~вводимые при этом новые предметные константы. Тогда 
    \[
        \Gamma, C_1(a_1), C_2(a_2), \dots, C_k(a_k) \vdash A.
    \]
    Применим теорему дедукции (что законно в силу п.3 в выводе с примененим правила C), получив в результате 
    \[
        \Gamma, C_1(a_1), C_2(a_2), \dots, C_{k - 1}(a_{k - 1}) \vdash C_k(a_k) \supset A.
    \]
    Заменим в полученном выводе все вхождения константы $a_k$ на переменную $z$, не встречающуюся в этом выводе, и получим
    \[
        \Gamma, C_1(a_1), C_2(a_2), \dots, C_{k - 1}(a_{k - 1}) \vdash C_k(z) \supset A.
    \]
    К полученному выводу применим правило обобщения и заключим
    \[
        \Gamma, C_1(a_1), C_2(a_2), \dots, C_{k - 1}(a_{k - 1}) \vdash \forall z\big(C_k(z) \supset A\big).
    \]
    Из леммы \ref{th:exists_theorems} (п.4) вытекает следующее: $\forall x(B \supset A) \vdash \exists xB \supset A$. Из этого заключаем, что
    \[
        \Gamma, C_1(a_1), C_2(a_2), \dots, C_{k - 1}(a_{k - 1}) \vdash \exists zC_k(z) \supset A
    \]
    Далее применяя правило переименования связанных переменных, получаем
    \[
        \Gamma, C_1(a_1), C_2(a_2), \dots, C_{k - 1}(a_{k - 1}) \vdash \exists y_kC_k(y_k) \supset A.
    \]
    Затем учтём, что поскольку формула $\exists y_kC_k(y_k)$ уже встречалась в выводе, то 
    \[
        \Gamma, C_1(a_1), C_2(a_2), \dots, C_{k - 1}(a_{k - 1}) \vdash \exists y_kC_k(y_k),
    \]
    тогда по правилу MP и правилу сечения получаем
    \[
        \Gamma, C_1(a_1), C_2(a_2), \dots, C_{k - 1}(a_{k - 1}) \vdash A.
    \]
    По аналогии избавляемся от $C_{k - 1}(a_{k - 1}), C_{k - 2}(a_{k - 2}), \dots, C_1(a_1)$ и получаем требуемую выводимость: $\Gamma \vdash A$.
\end{proof}
\textbf{Пример}: $\vdash \forall x\big(A(x) \supset B(x)\big) \supset \big(\exists xA(x) \supset \exists xB(x)\big)$. \\
Вывод $\forall x\big(A(x) \supset B(x)\big), \exists xA(x) \vdash_C \exists xB(x)$:
\begin{enumerate}
    \item $\forall x\big(A(x) \supset B(x)\big)$ (гипотеза);
    \item $\exists xA(x)$ (гипотеза);
    \item $A(b)$ (из 2 по правилу C);
    \item $A(b) \supset B(b)$ (из 1 по правилу индивидуализации);
    \item $B(b)$ (из 3, 4 по MP);
    \item $\exists xB(x)$ (из 5 по правилу существования).
\end{enumerate}
Тогда по теореме \ref{th:c_rule_inference} имеет место
\[
    \forall x\big(A(x) \supset B(x)\big), \exists xA(x) \vdash \exists xB(x).
\]
Дважды применяя теорему дедукции, получаем требуемое:
\[
    \vdash \forall x\big(A(x) \supset B(x)\big) \supset \big(\exists xA(x) \supset \exists xB(x)\big).
\]