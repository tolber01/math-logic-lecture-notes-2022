\subsection{Теорема дедукции в теории ИП}
Теорема дедукции для пропозиционального ИВ не может быть перенесена без изменений на произвольные теории первого порядка. Однако некоторая ослабленная форма теоремы может быть доказана для них.

Пример: $A(x) \vdash \forall xA(x)$. Вывод состоит в применении к $A(x)$ правила Gen:
\begin{enumerate}
    \item $A(x)$ (гипотеза);
    \item $\forall xA(x)$ (из 1 по Gen).
\end{enumerate}
Рассмотрим теперь интерпретацию $D = \{a, b\}$. Предикат $A$ возьмём таким, что $A(a)$ истинно, а $A(b)$ ложно. Тогда формула $A(x) \supset \forall xA(x)$ не является верной, если положить свободную переменную $x = a$. Действительно, ясно, что $\forall xA(x)$ ложно, а с другой стороны, взяв значение свободной переменной $x = a$, получим, что посылка импликации будет истинной ($A(a)$ истинно), что влечёт ложность импликации.

Пусть $A$ --- формула из множества формул $\Gamma$, а $B_1, B_2, \dots, B_n$ --- некоторый вывод из $\Gamma$. Будем говорить, что формула $B_i$ \textit{зависисит от $A$ в этом выводе}, если для этой формулы выполнено какое-нибудь из условий:
\begin{enumerate}
    \item формула $B_i$ есть $A$;
    \item формула $B_i$ есть непосредственное следствие по одному из правил вывода некоторых предшествующих в этом выводе формул, из которых хотя бы одна зависит от $A$.
\end{enumerate}

\begin{lemma}
    Если формула $B$ не зависит от формулы $A$ в выводе $\Gamma, A \vdash B$, то $\Gamma \vdash B$.
\end{lemma}
\begin{proof}
    Пусть $B_1, B_2, \dots, B_n$ --- произвольный вывод $B$ из $\Gamma \cup \{A\}$, в котором $B$ независит от $A$. Ясно, что $B_n$ совпадает с $B$.

    Докажем по индукции по $n \in \mathbb{N}$ (то есть по длине вывода), что $\Gamma \vdash B_n$, то есть $\Gamma \vdash B$. Так мы докажем, что утверждение леммы верно, каков бы ни был вывод $\Gamma, A \vdash B$.

    \textbf{База}: $n = 1$. Вывод состоит всего из одной формулы --- формулы $B$. Ясно, что $B$ в таком случае не совпадает с $A$, поскольку в противном случае имела бы место зависимость $B$ от $A$, противоречащая условию леммы. В таком случае очевидно, что написанный вывод, состоящий из одной формулы, по совместительству является ещё и выводом $\Gamma \vdash B$, что доказывает базу индукции.

    \textbf{Индукционное предположение}: пусть утверждение доказано для всех выводов $\Gamma, A \vdash B$ всех длин $k \leqslant n - 1$.

    \textbf{Индукционный шаг}: докажем, что утверждение будет верно для вывода и длины $n$. Очевидно, что утверждение леммы будет верным, если окажется, что $B$ входит в $\Gamma$ или является одной из аксиом. Пусть теперь $B_n$ в нашем произвольном выводе получено из каких-то предшествующих формул $B_i$ и $B_j$ ($i, j < n$) по одному из двух правил вывода. Ясно, что поскольку $B_n$ (то есть $B$) не зависит от $A$, то ни одна из формул $B_i$ и $B_j$ не зависит от $A$. Тогда согласно индукционному предположению и произвольности формулы $B$ в формулировке леммы, получаем, что имеют место выводы $\Gamma \vdash B_i$ и $\Gamma \vdash B_j$. Тогда по правилу сечения извлекаем искомый вывод: $\Gamma \vdash B_n$.
\end{proof}

\begin{theorem}[\textit{дедукция в ИП}]
    Пусть $\Gamma, A \vdash B$ и существует вывод $B$ из $\Gamma \cup \{A\}$, в котором ни при каком применении правила Gen к формулам, зависящим в этом выводе от $A$, не связывается квантором никакая свободная переменная формулы $A$. Тогда $\Gamma \vdash A \supset B$.
\end{theorem}

\begin{corollary}
    Если $\Gamma, A \vdash B$ и существует вывод $B$ из $\Gamma, A$, построенный без применения правила Gen к свободным переменным формулы $A$, то $\Gamma \vdash A \supset B$.
\end{corollary}

\begin{corollary}
    Если формула $A$ замкнута и $\Gamma, A \vdash B$, то $\Gamma \vdash A \supset B$.
\end{corollary}