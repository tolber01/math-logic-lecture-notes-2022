\subsection{Кванторы и формулы логики предикатов первого порядка}
Логика предикатов первого порядка представляет собой обобщение логики высказываний на такие логические рассуждения, которые не могут быть формализованы средствами последней.

Классический пример: все люди смертны, Сократ --- человек, следовательно, Сократ смертен.

Пусть $P(x)$ означает, что объект $x$ обладает свойством $P$ ($P$ --- \textit{предикат}). Посредством $\forall xP(x)$ будем обозначать утверждение <<для всякого $x$ выполнено свойство $P$>>, $\exists xP(x)$ --- утверждение <<существует $x$, для которого выполнено свойство $P$>>. Символ $\forall$ называется \textit{квантором всеобщности}, а $\exists$ --- \textit{квантором существования}.

Пусть константный символ $S$ означает Сократа, предикат $M(x)$ --- <<$x$ смертен>>, $H(x)$ --- <<$x$ --- человек>>. Тогда приведённое выше рассуждение формализуется так:
\[
    \frac{\forall x\big(H(x) \supset M(x)\big),\ H(S)}{M(S)}.
\]

Ещё пример: все боятся Дракулы, Дракула боится только меня, следовательно, я Дракула. \\
$D$ --- Дракула, $A(x, y)$ --- <<$x$ боится $y$>>, $I$ --- я, $E(x, y)$ --- <<$x$ есть $y$>>. Тогда указанное выше утверждение формализуется так:
\[
    \frac{\forall xA(x, D),\ \forall y\big(A(D, y) \supset E(y, I)\big)}{E(I, D)}
\]

Алфавит теории ИП включает следующие символы:
\begin{enumerate}
    \item предметные (индивидные) переменные: $x_1, x_2, \dots$;
    \item предметные (индивидные) константы: $a_1, a_2, \dots$;
    \item предикатные буквы $A_1^1, A_1^2, \dots, A_1^j, \dots$;
    \item функциональные буквы $f_1^1, f_1^2, \dots, f_1^j, \dots$;
    \item пропозициональные связки $\neg$, $\supset$;
    \item квантор всеобщности $\forall$;
    \item квантор существования $\exists$;
    \item специальные символы $($, $)$, $,$.
\end{enumerate}
Верхний индекс предикатных и функциональных букв указывает число аргументов буквы, а нижний служит номером соответствующей буквы.

\begin{definition*}
    \textit{Термом} называется слово в алфавите ИП, построенное по правилам:
    \begin{enumerate}
        \item Всякая предметная переменная или предметная константа --- терм;
        \item Если $f_i^j$ есть функциональная буква, а $t_1, t_2, \dots, t_j$ --- термы, то $f_i^j(t_1, t_2, \dots, t_j)$ --- терм;
        \item Слово является термом тогда и только тогда, когда оно может быть получено по правилам 1, 2.
    \end{enumerate}
\end{definition*}

\begin{definition*}
    Слово в алфавите ИП называется \textit{элементарной формулой}, если оно имеет такой и только такой вид: $A_i^n(t_1, t_2, \dots, t_n)$, где $A_i^n$ --- предикатный символ арности $n$, а $t_1, t_2, \dots, t_n$ --- термы.
\end{definition*}

\begin{definition*}
    Формулой теории ИП называется слово в алфавите ИП построенное по правилам:
    \begin{enumerate}
        \item Всякая элементарная формула является формулой;
        \item Если $A$ и $B$ --- формулы, а $y$ --- предметная переменная, то $(\neg A)$, $(A \supset B)$, $(\forall yA)$ также являются формулами. В последнем случае формула $A$ называется \textit{областью действия квантора} $\forall$.
        \item Слово является формулой тогда и только тогда, когда оно может быть получено по правилам 1, 2. 
    \end{enumerate}
\end{definition*}

\begin{remark*}\leavevmode
    \begin{enumerate}
        \item Связки $\equiv$, $\land$, $\lor$ могут использоваться как сокращения по эквивалентностям
        \begin{gather*}
            A \land B \Longleftrightarrow \neg(A \supset \neg B); \\
            A \lor B \Longleftrightarrow \neg A \supset B; \\
            A \equiv B \Longleftrightarrow (A \supset B)\land(B \supset A).
        \end{gather*}

        \item Квантор $\exists$ можно использовать для сокращения по эквивалентности $\exists xA \Longleftrightarrow \neg\big(\forall x(\neg A)\big)$.

        \item Можно опускать лишние скобки по тем же правилам, что и в ИВ с учётом того, что кванторы по приоритету находятся между $\supset$ и $\lor$ (сильнее $\supset$, но слабее $\lor$).

        \item Можно опускать скобки в (под)формулах вида $Q_1\Big(Q_2\big(\dots Q_{n - 1}(Q_nA)\dots\big)\Big)$, где $Q_i$, $i = 1, 2, \dots, n$, есть кванторы, то есть допустимо писать $Q_1Q_2\dots Q_{n - 1}Q_nA$.
    \end{enumerate}
\end{remark*}

\begin{definition*}
    Вхождение переменной $x$ в формулу называется \textit{связанным}, если $x$ --- переменная входящего в эту формулу квантора $\forall x$ или находится в области действия входящего в эту формулу квантора $\forall x$. В противном случае вхождение переменной $x$ в формулу называется \textit{свободным}.
\end{definition*}

\begin{definition*}
    Формула $x$ называется \textit{свободной} (\textit{связанной}) в формуле, если существует хотя бы одно её свободное (связанное) вхождение в эту формулу.
\end{definition*}
Вполне возможно, что одна и та же переменная может быть как связанной, так и свободной в одной и той же формуле.

\begin{definition*}
    Формула называется \textit{замкнутой}, если она не содержит свободных переменных.
\end{definition*}