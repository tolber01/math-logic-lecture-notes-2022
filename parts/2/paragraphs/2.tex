\subsection{Интерпретации и метод моделей}
\begin{definition*}
    \textit{Интерпретацией} называется непустое множество $D$ (\textit{область интерпретации}) и соответствие, относящее каждой предикатной букве $A_i^n$ некоторое $n$-арное отношение в $D$, каждой функциональной букве $f_i^n$ --- некоторую $n$-местную функцию $D^n \to D$, каждой предметной константе --- некоторый элемент $D$. Предметные переменные принимают значения в $D$, связки $\neg$, $\supset$ и квантор $\forall$ имеют обычный смысл.
\end{definition*}
При заданной интерпретации замкнутая формула переходит в истинное или ложное высказывание, формула со свободными переменными --- в некоторое отношение на $D$, которое может быть выполнено для одних значений переменных из $D$ и не выполнено для других.
\begin{definition*}
    Формула называется \textit{истинной в данной интерпретации}, если она \textit{выполнена} для всех элементов из $D$.
\end{definition*}
\begin{definition*}
    Формула называется \textit{ложной в данной интерпретации}, если она \textit{не выполнена} ни для одного элемента из $D$.
\end{definition*}
\begin{definition*}
    Интерпретация называется \textit{моделью} для множество формул $\Gamma$, если каждая формула из $\Gamma$ истинна в ней.
\end{definition*}
\begin{definition*}
    Формула называется \textit{логически общезначимой}, если она истинна в любой интерпретации.
\end{definition*}
Логически общезначимые формулы также называются \textit{логическими законами}.
\begin{definition*}
    Формула $A$ называется \textit{противоречием}, если $\neg A$ логически общезначима.
\end{definition*}
\begin{definition*}
    Формула называется \textit{выполнимой}, если существует интерпретация, в которой она выполнена хотя бы на одном подмножестве элементов из $D$.
\end{definition*}
Метод моделей установления общезначимости (противоречивости, выполнимости) формула заключается в подборе моделей (интерпретаций) или доказательстве общезначимости (противоречивости) с помощью интерпретаций.