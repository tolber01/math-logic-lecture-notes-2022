\subsection{Подходы к формализации алгоритма}
Для формализации понятия алгоритма был выбран следующий подход. Выбирается конечный набор исходных объектов (элементов) и конечный набор правил построения из них новых объектов. Это означает, что определяется:
\begin{enumerate}[label=\arabic*)]
    \item алфавит данных и алфавит алгоритмов (конечные наборы символов, которыми записываются обрабатываемые данные и команды алгоритма);
    \item наборы объектов, над которыми производятся операции~---~слова в алфавите данных;
    \item конечный набор элементарных операций;
    \item ограничения на используемую память.
\end{enumerate}
В зависимости от выбора этих атрибутов получается конкретная алгоритмическая модель. В теории алгоритмов сформировались три типа алгоритмических моделей.

Исторически \underline{первой} была разработана формализация алгоритмов с помощью \textit{рекурсивных числовых функций}. Теория рекурсивных функций была разработана Куртом Гёделем и Жаком Эрбраном. Поэтому этот подход к формализации алгоритмической вычислимости называется \textit{вычислимость по Эрбрану--Гёделю}.

\underline{Основной результат этой теории}: класс вычислимых с помощью алгоритмов в широком интуитивном смысле числовых функций совпадает с классом частично рекурсивных функций (тезис Чёрча). Иными словами, любая функция, которую можно вычислить известными человечеству методами, допускает представление через частично рекурсивные функции. Это утверждение называется тезисом, потому что его невозможно доказать в силу принципиальной неформализуемости понятия алгоритма.

Тезис Чёрча~---~это естественнонаучный факт, подтверждаемый опытом, накопленным в математике за всю её историю. Его можно опровергнуть, построив вычислимую функцию, не представимую с помощью частично рекурсивных функций.

\underline{Второй подход} к формализации вычислимости связан с абстрактными вычислительными машинами~---~детерминированными
устройствами, способными выполнять в каждый момент времени лишь примитивные операции. Первая машина была построена Аланом Тьюрингом (\textit{машина Тьюринга}). Затем Эмиль Пост предложил свою машину (\textit{машина Поста}), но позже было доказано, что машины Тьюринга и Поста эквивалентны, т.е. любая числовая функция, вычислимая машиной Тьюринга, вычисляется также и машиной Поста и наоборот.

\underline{Основной результат этой теории}: класс вычислимых с помощью алгоритмов в широком интуитивном смысле числовых функций совпадает с классом вычислимых по Тьюрингу функций (\textit{тезис Тьюринга}). Он также недоказуем, как тезис Чёрча. Впоследствии было доказано, что вычислимости по Эрбрану-Гёделю и Тьюрингу эквивалентны, поэтому тезис Тьюринга также называют \textit{тезисом Чёрча--Тьюринга}.

\underline{Третий подход} связан с преобразованиями слов в произвольных алфавитах с элементарными операциями подстановки. Алгоритмы преобразования слов были предложены Андреем Андреевичем Марковым (\textit{нормальные алгорифмы Маркова}). Преимущество этого подхода в максимальной абстракции и применимости к объектам любой природы (не обязательно к числам).

Существование трёх различных подходов к формализации алгоритма не ведёт к утрате его универсальности. Доказано, что они сводятся друг к другу и в конечном счёте приводят к понятию вычислимой числовой функции~---~функции, для которой существует эффективный алгоритм вычисления. Это подтверждает тезис о единстве материального мира.