\subsection{Понятие алгоритма}
\textit{Алгоритм} (от имени хорезмского учёного Абу Абдуллаха Мухаммеда ибн Мусы аль-Хорезми)~---~набор инструкций, описывающих порядок действий исполнителя для решения задачи.

\textbf{Требования к (вычислительным) алгоритмам}:
\begin{itemize}
    \item приспособленность к работе с данными;
    \item \textit{конечность}: число шагов должно быть конечно;
    \item \textit{элементарность}: шаги алгоритма должны быть детализированы;
    \item \textit{детерминированность}: после каждого шага точно определён следующий или определено окончание работы алгоритма;
    \item \textit{направленность}: все шаги направлены на решение задачи;
    \item \textit{результативность}: остановка после конечного числа шагов с указанием того, что считать результатом;
    \item \textit{сходимость}: получение результата за конечное число шагов для набора исходных данных из некоторого множества (область сходимости);
    \item \textit{конечность описания}: описание и механизм реализации (средства пуска, реализации элементарных шагов, остановки, выдачи результатов) конечны;
    \item \textit{массовость}: приспособленность к решению классов задач при любых исходных данных из некоторого класса.
\end{itemize}

Итак, конкретизируем понятие алгоритма. \\
\textit{Алгоритм}~---~предназначенная для решения некоторого класса задач конечная последовательность элементарных инструкций, каждая из которых имеет чёткий смысл и может быть выполнена с конечными вычислительными затратами.

При любых входных данных из области сходимости алгоритм завершается после конечного числа шагов и выдаёт результат (выходные данные).

Такое понимание алгоритма применялось математиками (и не только ими) в течение тысячелетий. Но на рубеже XIX--XX веков возникла потребность в строгой математической формализации алгоритма. Связано это было с пересмотром (вернее, строгим построением) оснований математики.

Кроме того, пересмотр оснований математики был связан с наличием большого числа логических парадоксов, возникших в связи с несовершенностью имеющихся на тот момент идей и принципов построения математических теорий. Известны, например, следующие парадоксы:
\begin{itemize}
    \item \textit{Парадокс лжеца}: является ли верным утверждение <<Я лгу>>?
    \item \textit{Парадокс Рассела}: пусть определено множество $X = \{Y\ |\ Y \not\in Y\}$, спрашивается, верно ли, что $X \in X$? (парадокс, возникший в наивной теории множеств Кантора). \\
    \textit{Парадокс брадобрея} на ту же тему: в некоторой деревне живёт брадобрей, которому велено брить тех и только тех жителей деревни, кто не бреется сам, спрашивается, будет ли брадобрей брить сам себя?
    \item \textit{Парадокс Банаха--Тарского} (или \textit{парадокс удвоения шара}) говорит, что трёхмерный шар равносоставлен двум своим копиям. Более общо: любые два ограниченных подмножества Евклидова пространства с непустой внутренностью являются равносоставленными. Это парадокс, являющийся следствием знаменитой \textit{аксиомы выбора}, входящей в аксиоматику ZFC (Zermelo--Fraenkel + axiom of Choice) Цермело--Френкеля с аксиомой выбора.
\end{itemize}