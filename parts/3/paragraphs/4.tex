\subsection{Рекурсивные функции}
\subsubsection{Операторы суперпозиции, примитивной рекурсии и минимизации}
Рассматриваются только частичные числовые функции.
\begin{definition*}
    Функция $F(\widetilde x^n) = f\big(g_1(\widetilde x^n), \dots, g_m(\widetilde x^n)\big)$ называется \textit{суперпозицией функций} $f$ и $g_1, \dots, g_m$, причём функция $F$ определена на наборе $\widetilde\alpha^n$ тогда и только тогда, когда $g_i$ определены на $\widetilde\alpha^n$ для всех $i = 1, 2, \dots, m$ и $f$ определена на наборе $\langle g_1(\widetilde\alpha^n), \dots, g_m(\widetilde \alpha^n)\rangle$. Тогда
    \[
        F(\widetilde\alpha^n) = f\big(g_1(\widetilde\alpha^n), \dots, g_m(\widetilde\alpha^n)\big).
    \]
\end{definition*}
Оператор суперпозиции обозначается $S(f^m\colon g_1^n, \dots, g_m^n)$.
\begin{definition*}
    Функция $f(\widetilde x^n)$ получается из функций $g(\widetilde x^{n - 1})$ и $h(\widetilde x^{n + 1})$, $n \geqslant 1$, с помощью \textit{оператора примитивной рекурсии}, если она может быть задана \textit{схемой примитивной рекурсии}:
    \[
        \begin{cases}
            f(x_1, \dots, x_{n - 1}, 0) = g(\widetilde x^{n - 1}), \\
            f(x_1, \dots, x_{n - 1}, y + 1) = h\big(x_1, \dots, x_{n - 1}, y, f(x_1, \dots, x_{n - 1}, y)\big),\ y \geqslant 0.
        \end{cases}
    \]
\end{definition*}
Оператор примитивной рекурсии по переменной $x_n$ обозначается $R_{x_n}(g, h)$. \\
При $n = 1$ (функция одной переменной) схема примитивной рекурсии имеет вид:
\[
    \begin{cases}
        f(0) = a, \\
        f(y + 1) = h\big(y, f(y)\big),
    \end{cases}
\]
где $a \in \mathbb{N}_0$~---~постоянная.

\begin{definition*}
    Функция $f(\widetilde x^n)$ получается из функции $g(\widetilde x^{n + 1})$, $n \geqslant 1$, с помощью \textit{оператора минимизации}, если для произвольного набора $\widetilde\alpha^n$ $f(\widetilde\alpha^n)$ определена и равна $y$ тогда и только тогда, когда $g(\alpha_1, \dots, \alpha_n, 0)$, \dots, $g(\alpha_1, \dots, \alpha_n, y - 1)$ определены и не равны нулю, а $g(\alpha_1, \dots, \alpha_n, y)$ определена и равна нулю.
\end{definition*}
Оператор минимизации по переменной $y$ обозначает $\mu_y\!\left[g^{n + 1}(x_1, \dots, x_n, y) = 0\right]$.

Операторы примитивной рекурсии можно применять по любым переменным, от которых зависят функции $f$, $g$ и $h$ (не только по последним), но всегда надо указывать, по каким переменным они применяются. 

\subsubsection{Классы рекурсивных функций. Тезис Чёрча}
\begin{definition*}
    \textit{Простейшими} называются следующие функции:
    \begin{enumerate}
        \item $s(x) = x + 1$ (\textit{функция следования});
        \item $o(x) = 0$ (\textit{нулевая функция});
        \item $I_m^n(\widetilde x^n) = x_m$, $1 \leqslant m \leqslant n$, $n \geqslant 1$ (\textit{селекторная функция}, или \textit{функция выбора аргумента}).
    \end{enumerate}
\end{definition*}

\begin{definition*}
    Функция называется \textit{примитивно рекурсивной}, если она может быть получена из простейших функций конечным числом применений операторов суперпозиции и примитивной рекурсии.
\end{definition*}
Класс всех примитивно рекурсивных функций обозначается $K_{\text{пр}}$.

\begin{definition*}
    Функция называется \textit{частично рекурсивной}, если она может быть получена из простейших функций конечным числом применений операторов суперпозиции, примитивной рекурсии и минимизации. 
\end{definition*}
Класс всех частично рекурсивных функций обозначается $K_{\text{чр}}$.

\begin{definition*}
    Частично рекурсивная функция называется \textit{общерекурсивной} (\textit{вычислимой}), если она всюду определена.
\end{definition*}
Класс всех общерекурсивных функций обозначается $K_{\text{ор}}$.

Очевидно, что $K_{\text{ор}} \subset K_{\text{чр}}$; доказано, что $K_{\text{пр}} \subset K_{\text{ор}}$ (включения строгие). Таким образом, $K_{\text{пр}} \subset K_{\text{ор}} \subset K_{\text{чр}}$.

\textbf{Тезис Чёрча}. Класс всех вычислимых частичных числовых функций совпадает с $K_{\text{чр}}$. В более общей формулировке: любую эффективно вычислимую функцию можно представить в виде частично рекурсивной частичной числовой функции.

\textbf{Примеры}.
\begin{enumerate}
    \item Доказать, что функция $f(x) = x + n$, где $n \in \mathbb{N}_0$~---~постоянная, примитивно рекурсивна. \\
    \underline{Решение}. \\
    Запишем $f(x)$ в следующем виде:
    \[
        f(x) = \Big(\dots\big((x + 1) + 1\big) \dots + 1\Big) + 1,
    \]
    где количество добавлений единицы есть $n$. Тогда $f$ можно записать с помощью суперпозиции простейших функций:
    \[
        f(x) = s\Big(s\big(\dots s(x) \dots\big)\Big),
    \]
    где функция $s$ применена ровно $n$ раз. Такое представление доказывает примитивную рекурсивность функции $f$.
    \item Доказать, что функция $f(x, y) = x + y$ примитивно рекурсивна. \\
    \underline{Решение}. \\
    Запишем схему примитивной рекурсии для функции $f$ по переменной $y$:
    \[
        \begin{cases}
            f(x, 0) = I_1^1(x), \\
            f(x, y + 1) = s\Big(I_3^3\big(x, y, f(x, y)\big)\Big).
        \end{cases}
    \]
    В упомянутой схеме примитивной рекурсии полагается $g(x) = I_1^1(x)$ и $h(x, y, z) = s\big(I_3^3(x, y, z)\big)$. Полученное представление доказывает примитивную рекурсивность функции $f$.
\end{enumerate}