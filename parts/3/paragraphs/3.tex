\subsection{Машина Тьюринга}
\subsubsection{Устройство машины Тьюринга}
Машина Тьюринга представляет собой \textit{абстрактное} устройство, состоящее из бесконечной ленты, считывающей (и печатающей) головки и управляющего устройства.

Лента разбита на ячейки (клетки); в каждой из них в произвольный дискретный момент времени находится ровно один символ \textit{внешнего алфавита} $A = \{a_0, a_1, \dots, a_{n - 1}\}$, $n \geqslant 2$. Алфавит содержит символ, называемый \textit{пустым} (обычно это $a_0$, и за него принимается $0$ (нуль)). Клетка с пустым символом называется \textit{пустой}. Бесконечность ленты означает, что в любой момент времени она конечна, но число клеток всегда можно увеличить в обе стороны насколько необходимо (потенциальная бесконечность).

Управляющее устройство в каждый момент времени находится в некотором состоянии $q_j$ из внутреннего алфавита $Q = \{q_0, q_1, \dots, q_{r - 1}\}$, $r \geqslant 2$. Иногда в $Q$ выделяются непересекающиеся подмножества $Q_0$ и $Q_1$ \textit{заключительных} и \textit{начальных состояний соответственно}. В дальнейшем $Q_0 = \{q_0\}$, $Q_1 = \{q_1\}$.

\begin{figure}[H]
    \centering
    \begin{tikzpicture}[every node/.style={block}, block/.style={minimum height=1.5em,outer sep=0pt,draw,rectangle,node distance=0pt}]
        \node (A)[minimum size=3em, very thick] {};
        \node (B)[left=of A, minimum size=3em] {};
        \node (C)[left=of B, minimum size=3em] {};
        \node (D)[right=of A, minimum size=3em] {};
        \node (ALabel)[draw=none,above=1.1em of D] {Просматриваемый символ};
        \draw[-] (ALabel) -- (A.center);
        \node (E)[right=of D, minimum size=3em] {};
        \node (F)[right=of E, minimum size=3em] {};
        \node (G)[right=of F, minimum size=3em] {};
        \node (rightest)[right=of G, minimum size=3em, draw=none] {\dots};
        \node (leftest)[left=of C, minimum size=3em, draw=none] {\dots};
        \node (H) [below=3em of A, draw=black, very thick, minimum size=3em] {};
        \begin{scope}[transform canvas={xshift=-1.25em}]
            \draw[-stealth] (A) -- node[left, text width=2cm, align=right, draw=none] {Просмотр символа} (H);
        \end{scope}
        \begin{scope}[transform canvas={xshift=+1.25em}]
            \draw[-stealth] (H) --node[right, text width=1.5cm, align=left, draw=none] {Замена символа} (A);
        \end{scope}
        \draw[stealth-stealth] ([yshift=-0.5em]H.south west) -- node[below, draw=none] {Перемещение указателя} ([yshift=-0.5em]H.south east);
    \end{tikzpicture}
\end{figure}
Головка перемещается вдоль ленты так, что в каждый момент времени она обозревает ровно одну ячейку, считывает содержимое обозреваемой ячейки, стирает находящийся в ней символ, записывает другой символ внешнего алфавита $A$ (он может совпадать с прежним).

Управляющее устройство в зависимости от состояния, в котором оно находится, и обозреваемого головкой символа изменяет своё внутреннее состояние или остаётся в прежнем, выдаёт головке команду напечатать в ячейке определённый символ $A$ и сдвинуться в следующий момент времени на одну клетку влево, вправо или остаться на месте.

Таким образом, работа управляющего устройства описывается тремя функциями:
\[
    G\colon Q\times A \to Q; \quad F\colon Q\times A \to A; \quad D\colon Q\times A \to \{\text{S}, \text{L}, \text{R}\}.
\]
Символы S, L, R означают оставление головки на месте, движение на клетку влево и вправо соответственно. Функции $G$, $F$ и $D$ называются \textit{функциями переходов}, \textit{выходов} и \textit{движения} соответственно. Их можно записать пятёркой вида $q_ia_jq_{ij}a_{ij}d_{ij}$, где $q_{ij} = G(q_i, a_j)$, $a_{ij} = F(q_i, a_j)$, $d_{ij} = D(q_i, a_j)$. Они называются \textit{командами} машины Тьюринга. В общем случае команды определены не для каждой пары $\langle q_i, a_j\rangle$.

Список всех команд, определяющих работу машины Тьюринга, называется её \textit{программой}. Программу можно записывать в таблицу. Если для пары $\langle q_i, a_j\rangle$ команда отсутствует, то в таблице ставится прочерк в соответствующей ячейке.
\begin{table}[H]
    \centering
    \begin{tabular}{| c | c | c | c | c | c |}
        \hline  & \HC $q_0$ & \HC \dots & \HC $q_i$ & \HC \dots & \HC $q_{r - 1}$ \\
        \hline \HC $a_0$ & --- & \dots & --- & \dots & --- \\
        \hline \HC $\vdots$ & $\vdots$ & $\ddots$ & $\vdots$ & $\ddots$ & $\vdots$ \\
        \hline \HC$a_j$ & --- & \dots & $q_{ij}\ a_{ij}\ d_{ij}$ & \dots & --- \\
        \hline \HC $\vdots$ & $\vdots$ & $\ddots$ & $\vdots$ & $\ddots$ & $\vdots$ \\
        \hline \HC $a_{n - 1}$ & --- & \dots & --- & \dots & ---\\
        \hline
    \end{tabular}
\end{table}

\subsubsection{Конфигурации}
Работу машины Тьюринга будем описывать на языке конфигураций.

Пусть в момент времени $t$ самая левая непустая ячейка $C_1$ на ленте содержит символ $a_{j_1}$, а самая правая непустая ячейка $C_s$~---~символ $a_{j_s}$ ($s \geqslant 2$). Тогда будем говорить, что в момент времени $t$ на ленте записано слово $P = a_{j_1}\dots a_{j_s}$. \\
При $s = 1$ на ленте записано слово $P = a_{j_1}$. Пустыми считаются клетки, содержащие пустой символ $a_0 \in A$. Далее пустой символ будем обозначать $\Lambda$.

Пусть в момент времени $t$ управляющее устройство находится в состоянии $q_i$, а головка обозревает символ $a_{j_\ell}$ слова $P$ ($\ell \geqslant 2$). Тогда слово
\[
    a_{j_1}\dots a_{j_{\ell - 1}}q_ia_{j_\ell}\dots a_{j_s}
\]
называется \textit{конфигурацией машины в момент времени $t$}.

При $\ell = 1$ конфигурация имеет вид $q_iP$. Если головка обозревает пустую ячейку, находящуюся слева (справа) от слова $P$, и между ней и первым (последним) непустым символом слова находятся $\nu \geqslant 0$ пустых ячеек, то конфигурацией называется слово
\[
    q_i\Lambda^{\nu + 1}P\ \ (P\Lambda^\nu q_i\Lambda),
\]
где $a^\nu$~---~сокращённая запись слова, состоящего из $\nu$ символов $a$. Если в момент $t$ лента пуста, то конфигурацией будет слово $q_i\Lambda$.

Пусть в момент $t$ машина имеет конфигурацию $a_{j_1}\dots a_{j_{\ell - 1}}q_ia_{j_\ell}\dots a_{j_s}$ и выполняется команда $q_ia_{j_\ell}q_{ij_\ell}a_{ij_\ell}d_{ij_\ell}$. Тогда при $d_{ij_\ell} = \text{L}$ в следующий момент машина будет иметь конфигурацию:
\begin{enumerate}
    \item $q_{ij_1}\Lambda a_{ij_1}a_{j_2}\dots a_{j_s}$, если $\ell = 1$;
    \item $q_{ij_2}a_{j_1}a_{ij_2}a_{j_3}\dots a_{j_s}$, если $\ell = 2$;
    \item $a_{j_1}\dots a_{j_{\ell - 2}}q_{ij_\ell}a_{j_{\ell - 1}}a_{ij_\ell}a_{j_{\ell + 1}}\dots a_{j_s}$, если $\ell > 2$.
\end{enumerate}
Если в какой-то момент времени управляющее устройство приходит в заключительное состояние $q_0$, машина прекращает работу, и её конфигурация в этот момент называется \textit{заключительной}.

Если для пары $\langle q_i, a_j\rangle$ команда отсутствует, то машина прекращает работу и текущая конфигурация также считается заключительной. Конфигурация, соответствующая началу работы, называется \textit{начальной}.

Пусть $K$~---~конфигурация в некоторый момент времени, $K'$~---~конфигурация в следующий момент. Тогда говорят, что конфигурация $K'$ \textit{непосредственно выводима} из $K$ (обозначение $K \models K'$). Если $K_1$~---~начальная конфигурация, то последовательность $K_1, \dots, K_m$, где $K_i \models K_{i + 1}$, $1 \leqslant i < m$, называется \textit{тьюринговым вычислением}. При этом конфигурация $K_m$ \textit{выводима из конфигурации} $K_1$: $K_1 \vdash K_m$. Если к тому же $K_m$ является заключительной конфигурацией, то говорят, что $K_m$ \textit{заключительно выводима из} $K_1$: $K_1 \Vdash K_m$.

Слово на ленте в начальный момент времени называется \textit{исходным} (\textit{начальным}). Если $P_1$~---~исходное слово, то машина $T$, начав с него, либо остановится через конечное число шагов, либо никогда не остановится. В первом случае говорят, что машина $T$ \textit{применима к слову} $P_1$, и результатом применения является слово $P$, соответствующее заключительной конфигурации (обозначение $P = T(P_1)$). Во втором случае говорят, что машина $T$ \textit{не применима} к слову $P_1$.

В дальнейшем предполагаем, что:
\begin{enumerate}
    \item Исходное слово не пусто.
    \item В начальный момент головка находится на самой левой непустой ячейке ленты.
    \item Начальное состояние машины $q_1$.
    \item Внешний алфавит двоичный, т.е. $A = \{0, 1\}$, $0$~---~пустой символ. 
\end{enumerate}

\textit{Зоной работы} машины $T$ (на исходном слове $P_1$) называется множество всех ячеек, которые за время работы машина хотя бы один раз обозревала.

\subsubsection{Операции над машинами Тьюринга}
\begin{enumerate}
    \item \textit{Композиция машин Тьюринга}. \\
    Пусть машины $T_1$ и $T_2$ заданы программами $\Pi_1$ и $\Pi_2$ соответственно. Пусть при этом внутренние алфавиты этих машин не пересекаются, $q_0'$~---~некоторое заключительное состояние машины $T_1$, $q_1''$~---~некоторое начальное состояние машины $T_2$. Заменим всюду в программе $\Pi_1$ состояние $q_0'$ на $q_1''$ и объединим полученную программу с $\Pi_2$. Новая программа $\Pi$ задаёт машину $T$~---~\textit{композицию машин $T_1$ и $T_2$ по паре состояний $q_0'$, $q_1''$}. Композиция обозначается $T_1\circ T_2$ ($T_1T_2$), или подробнее $T = T\big(T_1, T_2, \langle q_0', q_1''\rangle\big)$. Внешний алфавит композиции $T_1T_2$ является объединением внешних алфавитов $T_1$ и $T_2$.

    \item \textit{Итерация машины Тьюринга}. \\
    Пусть $q'$~---~некоторое заключительное состояние машины $T$ с программой $\Pi$, $q''$~---~некоторое состояние машины $T$, не являющееся заключительным. Заменив всюду в программе $\Pi$ символ $q'$ на $q''$, получим программу $\Pi'$, определяющую машину $T'(q', q'')$~---~\textit{итерацию машины $T$ по паре состояний $q'$, $q''$}.

    \item \textit{Разветвление машин Тьюринга}. \\
    Пусть машины $T_1$, $T_2$, $T_3$ заданы программами $\Pi_1$, $\Pi_2$, $\Pi_3$ соответственно. Пусть при этом внутренние алфавиты этих машин попарно не пересекаются, $q_0'$, $q_0''$~---~некоторые различные заключительные состояния машины $T_1$. Заменим всюду в программе $\Pi_1$ состояние $q_0'$ некоторым начальным состоянием $q_1'$ машины $T_2$, $q_0''$~---~некоторым начальным состоянием $q_1''$ машины $T_3$. Новую программу объединим с $\Pi_2$ и $\Pi_3$. Получим программу $\Pi$ машины $T = T\big(T_1, \langle q_0', q_1'\rangle, T_2, \langle q_0'', q_1''\rangle, T_3\big)$~---~\textit{разветвления машин $T_2$ и $T_3$, управляемым машиной $T_1$}.
\end{enumerate}