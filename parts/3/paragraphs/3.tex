\subsection{Машина Тьюринга}
\subsubsection{Устройство машины Тьюринга}
Машина Тьюринга представляет собой \textit{абстрактное} устройство, состоящее из бесконечной ленты, считывающей (и печатающей) головки и управляющего устройства.

Лента разбита на ячейки (клетки); в каждой из клеток в произвольный дискретный момент времени находится ровно один символ \textit{внешнего алфавита} $A = \{a_0, a_1, \dots, a_{n - 1}\}$, $n \geqslant 2$. Алфавит содержит символ, называемый пустым