\subsection{Аксиоматическая теория исчисления высказываний}
\subsubsection{Определение теории исчисления высказываний}
Зададим теорию исчисления высказываний.

\textbf{Алфавит теории}:
\begin{enumerate}
    \item пропозициональные буквы $A, B, C, \dots, A_1, A_2, \dots$;
    \item пропозициональные связки $\neg, \supset$;
    \item специальные символы $($, $)$.
\end{enumerate}
\textit{Формулы} теории определяются рекуррентно по правилам:
\begin{enumerate}
    \item Любая пропозициональная буква есть формула.
    \item Если $\mathcal{A}$, $\mathcal{B}$ есть формулы, то слова $(\neg\mathcal{A})$, $(\mathcal{A} \supset \mathcal{B})$ также являются формулами.
    \item Слово является формулой в том и только том случае, когда оно получено по правилам 1 и 2. 
\end{enumerate}
Используются те же правила удаления лишних скобок (см. п. \ref{par:formulas_boolean}).

\textbf{Аксиомы теории}:
\begin{enumerate}[label=(А\arabic*)]
    \item $A \supset (B \supset A)$;
    \item $\Big(A \supset (B \supset C)\Big) \supset \Big((A \supset B) \supset (A \supset C)\Big)$;
    \item $(\neg B \supset \neg A) \supset \Big((\neg B \supset A) \supset B\Big)$.
\end{enumerate}
Единственное правило вывода: $B$ --- непосредственное следствие формул $A$, $A \supset B$, или
\[
    \frac{A,\ A \supset B}{B}.
\]
Это правило называется \textit{modus ponens} (\textit{MP}). Его правомерность обосновывается теоремой \ref{th:modus_ponens}.
\begin{remark}
    Аксиомы на самом деле являются \textit{схемами аксиом}. Это означает, что подстановка в схему любых формул вместо всех вхождений букв (все вхождения одной буквы заменяются одной и той же формулой) даёт аксиому в силу теоремы \ref{th:tautology_substitution}. Таким образом, множество аксиом в нашем исчислении бесконечно.
\end{remark}
\begin{remark}
    Остальные пропозициональные связки используются для сокращения формул по эквивалентным заменам:
    \begin{itemize}
        \item $A \land B \Longleftrightarrow \neg(A \supset \neg B)$;
        \item $A \lor B \Longleftrightarrow \neg A \supset B$;
        \item $A \equiv B \Longleftrightarrow (A \supset B) \land (B \supset A)$.
    \end{itemize}
\end{remark}

\subsubsection{Доказательства в исчислении высказываний}
\begin{lemma}\label{th:a_supset_a}
    $\vdash A \supset A$.
\end{lemma}
\begin{proof}\leavevmode
    \begin{enumerate}
        \item $\Big(A \supset \big((A \supset A) \supset A\big)\Big) \supset \Big(\big(A \supset (A \supset A)\big) \supset (A \supset A)\Big)$ (А2);
        \item $A \supset \big((A \supset A) \supset A\big)$ (А1);
        \item $\big(A \supset (A \supset A)\big) \supset (A \supset A)$ (из 1, 2 по MP);
        \item $A \supset (A \supset A)$ (А1);
        \item $A \supset A$ (из 3, 4 по MP).
    \end{enumerate}
\end{proof}

\begin{lemma}
    $\vdash (\neg A \supset A) \supset A$.
\end{lemma}
\begin{proof}\leavevmode
    \begin{enumerate}
        \item $\neg A \supset \neg A$ (лемма \ref{th:a_supset_a});
        \item $(\neg A \supset \neg A) \supset \big((\neg A \supset A) \supset A\big)$ (А3);
        \item $(\neg A \supset A) \supset A$ (из 1, 2 по MP).
    \end{enumerate}
\end{proof}

\begin{lemma}\label{th:implication_transitivity}
    $A \supset B,\ B \supset C \vdash A \supset C$.
\end{lemma}
\begin{proof}\leavevmode
    \begin{enumerate}
        \item $B \supset C$ (гипотеза);
        \item $(B \supset C) \supset \big(A \supset (B \supset C)\big)$ (А1);
        \item $A \supset (B \supset C)$ (из 1, 2 по MP);
        \item $A \supset B$ (гипотеза);
        \item $\big(A \supset (B \supset C)\big) \supset \big((A \supset B) \supset (A \supset C)\big)$ (А2);
        \item $(A \supset B) \supset (A \supset C)$ (из 3, 5 по MP);
        \item $A \supset C$ (из 4, 6 по MP).
    \end{enumerate}
\end{proof}

\begin{lemma}
    $\vdash A \supset \big(B \supset (A \supset B)\big)$.
\end{lemma}
\begin{proof}\leavevmode
    \begin{enumerate}
        \item $B \supset (A \supset B)$ (А1);
        \item $\big(B \supset (A \supset B)\big) \supset \Big(A \supset \big(B \supset (A \supset B)\big)\Big)$ (А1);
        \item $A \supset \big(B \supset (A \supset B)\big)$ (из 1, 2 по MP).
    \end{enumerate}
\end{proof}

\begin{lemma}\leavevmode
    \begin{enumerate}[label=\arabic*)]
        \item $\vdash A \supset (A \lor B)$;
        \item $\vdash B \supset (A \lor B)$.
    \end{enumerate}
\end{lemma}
\begin{proof}\leavevmode
    \begin{enumerate}[label=\arabic*)]
        \item ??? (\textit{можно вывести из $A, \neg A \vdash B$ и двойного применения теоремы дедукции}).
        \item Запись $B \supset (A \lor B)$ понимается как $B \supset (\neg A \supset B)$, что есть просто аксиома А1.
    \end{enumerate}
\end{proof}

\begin{lemma}\leavevmode
    \begin{enumerate}[label=\arabic*)]
        \item $A \land B \vdash A$;
        \item $A \land B \vdash B$.
    \end{enumerate}
\end{lemma}
\begin{proof}\leavevmode
    \begin{enumerate}[label=\arabic*)]
        \item ???.
        \item ???.
    \end{enumerate}
\end{proof}

\begin{lemma}\label{th:implication_and_middle}
    $A \supset (B \supset C),\ B \vdash A \supset C$
\end{lemma}
\begin{proof}\leavevmode
    \begin{enumerate}
        \item $A \supset (B \supset C)$ (гипотеза);
        \item $\big(A \supset (B \supset C)\big) \supset \big((A \supset B) \supset (A \supset C)\big)$ (А2);
        \item $(A \supset B) \supset (A \supset C)$ (из 1, 2 по MP);
        \item $B \supset (A \supset B)$ (А1);
        \item $B$ (гипотеза);
        \item $A \supset B$ (5, 4 по MP);
        \item $A \supset C$ (из 6, 3 по MP).
    \end{enumerate}
\end{proof}

\begin{lemma}\label{th:negations}\leavevmode
    \begin{enumerate}[label=\arabic*)]
        \item $\vdash \neg\neg A \supset A$;
        \item $\vdash A \supset \neg\neg A$;
        \item $\neg\neg A \vdash A$;
        \item $A \vdash \neg\neg A$.
    \end{enumerate}
\end{lemma}
\begin{proof}\leavevmode
    \begin{enumerate}[label=\arabic*)]
        \item\label{item:negnegA_A}
        \begin{enumerate}[label=\arabic*.]
            \item $(\neg A \supset \neg\neg A) \supset \big((\neg A \supset \neg A) \supset A\big)$ (А3);
            \item $\neg A \supset \neg A$ (лемма \ref{th:a_supset_a});
            \item $(\neg A \supset \neg\neg A) \supset A$ (из 1, 2 по лемме \ref{th:implication_and_middle});
            \item $\neg\neg A \supset (\neg A \supset \neg\neg A)$ (А1);
            \item $\neg\neg A \supset A$ (из 4, 3 по лемме \ref{th:implication_transitivity}).
        \end{enumerate}
        \item
        \begin{enumerate}[label=\arabic*.]
            \item $(\neg\neg\neg A \supset \neg A) \supset \big((\neg\neg\neg A \supset A) \supset \neg\neg A\big)$ (А3);
            \item $\neg\neg\neg A \supset \neg A$ (формула $\neg\neg A \supset A$ с подстановкой $A \leftrightharpoons \neg A$);
            \item $(\neg\neg\neg A \supset A) \supset \neg\neg A$ (из 2, 1 по MP);
            \item $A \supset (\neg\neg\neg A \supset A)$ (А1);
            \item $A \supset \neg\neg A$ (из 4, 3 по лемме \ref{th:implication_transitivity}).
        \end{enumerate}
        \item
        \begin{enumerate}[label=\arabic*.]
            \item $\neg\neg A$ (гипотеза);
            \item $\neg\neg A \supset A$ (выведенная теорема в п. 1);
            \item $A$ (из 1, 2 по MP).
        \end{enumerate}
        \item
        \begin{enumerate}[label=\arabic*.]
            \item $A$ (гипотеза);
            \item $A \supset \neg\neg A$ (выведенная теорема в п. 2);
            \item $\neg\neg A$ (из 1, 2 по MP).
        \end{enumerate}
    \end{enumerate}
\end{proof}

\subsubsection{Теорема дедукции}
\nobreak\vspace{0.5em}
\nopagebreak
\begin{theorem}[\textit{дедукция в ИВ}]
    Если $\Gamma$ --- множество формул и $\Gamma, A \vdash B$, то $\Gamma \vdash A \supset B$.
\end{theorem}
\begin{proof}\leavevmode

    Пусть $B_1, B_2, \dots, B_n$ --- вывод $\Gamma, A \vdash B$. Тогда $B_n$ совпадает с $B$. 

    Доказав индукцией по $i = 1, 2, \dots, n$, что $\Gamma \vdash A \supset B_i$, мы получим утверждение теоремы при $i = n$.

    \textbf{База}: $i = 1$ и $i = 2$. \\
    Проверим $\Gamma \vdash A \supset B_1$. Имеются следующие варианты: $B_1$ --- либо аксиома, либо совпадает с $A$, либо входит в $\Gamma$. Если $B_1$ совпадает с $A$, то $\Gamma \vdash A \supset A$ --- справедливое утверждение (лемма \ref{th:a_supset_a}). Если $B_1$ есть аксиома или $B_1 \in \Gamma$, то, написав первую аксиому $B_1 \supset (A \supset B_1)$, из $B_1$ и написанной аксиомы по MP выводим требуемую формулу $A \supset B_1$. \\
    Очевидно, что для формулы $B_2$ допускаются точно такие же варианты и рассуждения, ибо вывод $B_2$ по MP невозможен --- перед $B_2$ в выводе должно идти по меньшей мере две формулы.

    \textbf{Индукционное предположение}: Пусть $\Gamma \vdash A \supset B_j$ для всех $j = 1, 2, \dots, i - 1$.
    
    \textbf{Индукционный шаг}: Докажем $\Gamma \vdash A \supset B_i$. \\
    Снова имеются варианты: $B_i$ --- аксиома, либо $B_i$ совпадает с $A$, либо входит в $\Gamma$, либо выводится из каких-то предыдущих формул $B_r, B_q$ по MP ($r, q < i$). Проверка первых трёх случаев дословно повторяет проверку в базе индукции. Пусть $r < q$. Тогда $B_q$ имеет вид $B_r \supset B_i$. По предположению индукции верны $\Gamma \vdash A \supset B_r$ и $\Gamma \vdash A \supset B_q$. Запишем аксиому А2: $\big(A \supset (\underbrace{B_r \supset B_i}_{B_q})\big) \supset \big((A \supset B_r) \supset (A \supset B_i)\big)$. Дважды применяя MP ($A \supset B_q$ и написанная аксиома, а затем $A \supset B_r$ и $(A \supset B_r) \supset (A \supset B_i)$), получаем искомую формулу $A \supset B_i$.

    Таким образом, по индукции доказано, что имеет место $\Gamma \vdash A \supset B_n$, то есть $\Gamma \vdash A \supset B$.
\end{proof}

\begin{corollary*}
    Если $A \vdash B$, то $\vdash A \supset B$.
\end{corollary*}

\begin{remark*}[\textit{от автора конспекта}]
    Теорема дедукции верна и в обратную сторону: если $\Gamma \vdash A \supset B$, то справедливо $\Gamma, A \vdash B$.
\end{remark*}
\begin{proof}
    Напишем вывод $\Gamma \vdash A \supset B$, состоящий из формул $C_1, C_2, \dots, C_n$, где $C_n$ совпадает с $A \supset B$, а каждая из предыдущих формул либо аксиома, либо входит в $\Gamma$, либо выводится из некоторых предшествующих по MP. Далее допишем к написанному выводу формулу $A$ как гипотезу (а это именно гипотеза для $\Gamma, A \vdash B$) и выведем $B$ из $A$ и $C_n$ по MP.
\end{proof}

\subsubsection{Применение теоремы дедукции}
\begin{proof}[Доказательство леммы \ref{th:implication_transitivity}]\leavevmode
    
    Вывод $A \supset B,\ B \supset C,\ A \vdash C$:
    \begin{enumerate}
        \item $A \supset B$ (гипотеза);
        \item $A$ (гипотеза);
        \item $B$ (из 2, 1 по MP);
        \item $B \supset C$ (гипотеза);
        \item $C$ (из 3, 4 по MP).
    \end{enumerate}
    Теперь, применяя теорему дедукции, получаем $A \supset B, B \supset C \vdash A \supset C$.
\end{proof}

\begin{lemma}
    $A \supset (B \supset C) \vdash B \supset (A \supset C)$.
\end{lemma}
\begin{proof}\leavevmode

    Вывод $A \supset (B \supset C),\ B,\ A \vdash C$:
    \begin{enumerate}
        \item $A$ (гипотеза);
        \item $A \supset (B \supset C)$ (гипотеза);
        \item $B \supset C$ (из 1, 2 по MP);
        \item $B$ (гипотеза);
        \item $C$ (из 4, 3 по MP).
    \end{enumerate}
    Теперь, дважды применяя теорему дедукции, получаем требуемое: $A \supset (B \supset C) \vdash B \supset (A \supset C)$.
\end{proof}

\begin{lemma}\leavevmode
    \begin{enumerate}[label=\arabic*)]
        \item $\vdash \neg A \supset (A \supset B)$;
        \item $\vdash (\neg B \supset \neg A) \supset (A \supset B)$ (\textit{доказательство от противного});
        \item\label{item:contraposition} $\vdash (A \supset B) \supset (\neg B \supset \neg A)$ (\textit{контрапозиция});
        \item $\vdash A \supset \big(\neg B \supset \neg(A \supset B)\big)$;
        \item $\vdash (A \supset B) \supset \big((\neg A \supset B) \supset B\big)$.
    \end{enumerate}
\end{lemma}
\begin{proof}\leavevmode
    \begin{enumerate}[label=\arabic*)]
        \item Доказательство $\neg A,\ A \vdash B$:
        \begin{enumerate}[label=\arabic*.]
            \item $\neg A$ (гипотеза);
            \item $A$ (гипотеза);
            \item $A \supset (\neg B \supset A)$ (А1);
            \item $\neg A \supset (\neg B \supset \neg A)$ (А1);
            \item $\neg B \supset A$ (из 2, 3 по MP);
            \item $\neg B \supset \neg A$ (из 1, 4 по MP);
            \item $(\neg B \supset \neg A) \supset \big((\neg B \supset A) \supset B\big)$ (А3);
            \item $(\neg B \supset A) \supset B$ (из 6, 7 по MP);
            \item $B$ (из 5, 8 по MP).
        \end{enumerate}
        Теперь дважды применяем теорему дедукции и получаем искомую выводимость.

        \item Доказательство $\neg B \supset \neg A,\ A \supset B$:
        \begin{enumerate}[label=\arabic*.]
            \item $\neg B \supset \neg A$ (гипотеза);
            \item $A$ (гипотеза);
            \item $(\neg B \supset \neg A) \supset \big((\neg B \supset A) \supset B\big)$ (А3);
            \item $A \supset (\neg B \supset A)$ (А1);
            \item $(\neg B \supset A) \supset B$ (из 1, 3 по MP);
            \item $A \supset B$ (из 4, 5 по лемме \ref{th:implication_transitivity});
            \item $B$ (из 2, 6 по MP).
        \end{enumerate}
        Теперь дважды применяем теорему дедукции и получаем искомую выводимость.

        \item Доказательство $A \supset B,\ \neg\neg A \vdash B$:
        \begin{enumerate}[label=\arabic*.]
            \item $\neg\neg A$ (гипотеза);
            \item $A$ (согласно выведенному $\neg\neg A \vdash A$ в лемме \ref{th:negations});
            \item $A \supset B$ (гипотеза);
            \item $B$ (из 2, 3 по MP).
        \end{enumerate}
        Тогда по теореме дедукции справедливо $A \supset B \vdash \neg\neg A \supset B$. Далее докажем $A \supset B,\ \neg B \vdash \neg A$:
        \begin{enumerate}[label=\arabic*.]
            \item $\neg B \supset (\neg\neg A \supset \neg B)$ (А1);
            \item $\neg B$ (гипотеза);
            \item $\neg\neg A \supset \neg B$ (из 2, 1 по MP);
            \item $(\neg\neg A \supset \neg B) \supset \big((\neg\neg A \supset B) \supset \neg A\big)$ (А3);
            \item $(\neg\neg A \supset B) \supset \neg A$ (из 3, 4 по MP);
            \item $A \supset B$ (гипотеза);
            \item $\neg\neg A \supset B$ (согласно доказанному $A \supset B \vdash \neg\neg A \supset B$);
            \item $\neg A$ (из 7, 5 по MP).
        \end{enumerate}
        Тогда, дважды применяя теорему дедукции, получаем требуемую выводимость: $\vdash (A \supset B) \supset (\neg B \supset \neg A)$.

        \item Ясно, что $A,\ A \supset B \vdash B$. Дважды используя теорему дедукции, получаем $\vdash A \supset \big((A \supset B) \supset B\big)$. Далее согласно \ref{item:contraposition} получаем $\vdash \big((A \supset B) \supset B\big) \supset \big(\neg B \supset \neg \big(A \supset B)\big)$. Наконец, задействуя лемму \ref{th:implication_transitivity} на предыдущих двух выводимостях, получаем искомое $\vdash A \supset \big(\neg B \supset \neg(A \supset B)\big)$.

        \item Доказательство $A \supset B,\ \neg A \supset B \vdash B$:
        \begin{enumerate}[label=\arabic*.]
            \item $A \supset B$ (гипотеза);
            \item $\neg A \supset B$ (гипотеза);
            \item $(A \supset B) \supset (\neg B \supset \neg A)$ (доказанная формула контрапозиции);
            \item $\neg B \supset \neg A$ (из 1, 3 по MP);
            \item $(\neg A \supset B) \supset (\neg B \supset \neg\neg A)$ (доказанная формула контрапозиции);
            \item $\neg B \supset \neg\neg A$ (из 2, 5 по MP);
            \item $(\neg B \supset \neg\neg A) \supset \big((\neg B \supset \neg A) \supset B\big)$ (А3);
            \item $(\neg B \supset \neg A) \supset B$ (из 6, 7 по MP);
            \item $B$ (из 4, 8 по MP).
        \end{enumerate}
        Теперь дважды применяем теорему дедукции и получаем искомую выводимость.
    \end{enumerate}
\end{proof}