\subsection{Другие аксиоматические теории ИВ}
\subsubsection{Исчисление Гильберта--Аккермана}

\textbf{Примитивные связки}: $\neg$, $\lor$, а $A \supset B$ --- сокращение $\neg A \lor B$.

\textbf{Схемы аксиом}:
\begin{enumerate}[label=(А\arabic*)]
    \item $A \lor A \supset A$;
    \item $A \supset (A \lor B)$;
    \item $(A \lor B) \supset (B \lor A)$;
    \item $(B \supset C) \supset \big((A \lor B) \supset (A \lor C)\big)$.
\end{enumerate}

\textbf{Правило вывода}: MP.

\subsubsection{Исчисление Россера}
\textbf{Примитивные связки}: $\neg$, $\land$, а $A \supset B$ --- сокращение $\neg(A \land \neg B)$.

\textbf{Схемы аксиом}:
\begin{enumerate}[label=(А\arabic*)]
    \item $A \supset A \land A$;
    \item $A \land B \supset A$;
    \item $(A \supset B) \supset \big(\neg(B \land C) \supset \neg(C \land A)\big)$.
\end{enumerate}

\textbf{Правило вывода}: MP.

\subsubsection{Исчисление секвенций}
\textbf{Примитивные связки}: $\neg$, $\lor$, $\land$, $\supset$.

\textbf{Специальный символ}: $\vdash$.

\textit{Секвенциями} называются слова следующих трёх видов, где $A_1, A_2, \dots, A_n, B$ --- произвольные формулы:
\begin{enumerate}
    \item $A_1, A_2, \dots, A_n \vdash B$ (из $A_1, A_2, \dots, A_n$ следует $B_n$);
    \item $\vdash B$ ($B$ доказуема);
    \item $A_1, A_2, \dots, A_n \vdash$ (система $A_1, A_2, \dots, A_n$ противоречива).
\end{enumerate}

\textbf{Схема аксиом}: $A \vdash A$.

\textbf{Правила вывода}, где $\Gamma, \Gamma_1, \Gamma_2, \Gamma_3$ --- произвольные конечные (может быть, пустые) последовательности формул, а $A, B, C$ --- произвольные формулы:
\begin{enumerate}[label=\arabic*)]
    \item $\dfrac{\Gamma_1 \vdash A;\ \Gamma_2 \vdash B}{\Gamma_1, \Gamma_2 \vdash A \land B}$ (\textit{введение $\land$});
    \item $\dfrac{\Gamma \vdash A \land B}{\Gamma \vdash A}$ (\textit{удаление $\land$});
    \item $\dfrac{\Gamma \vdash A \land B}{\Gamma \vdash B}$ (\textit{удаление $\land$});
    \item $\dfrac{\Gamma \vdash A}{\Gamma \vdash A \lor B}$ (\textit{введение $\lor$});
    \item $\dfrac{\Gamma \vdash B}{\Gamma \vdash A \lor B}$ (\textit{введение $\lor$});
    \item $\dfrac{\Gamma_1 \vdash A \lor B;\ \Gamma_2, A \vdash C;\ \Gamma_3, B \vdash C}{\Gamma_1, \Gamma_2, \Gamma_3 \vdash C}$ (\textit{удаление $\lor$});
    \item $\dfrac{\Gamma, A \vdash B}{\Gamma \vdash A \supset B}$ (\textit{введение $\supset$});
    \item $\dfrac{\Gamma_1 \vdash A;\ \Gamma_2 \vdash A \supset B}{\Gamma_1, \Gamma_2 \vdash B}$ (\textit{удаление $\supset$});
    \item $\dfrac{\Gamma, A \vdash}{\Gamma \vdash \neg A}$ (\textit{введение $\neg$});
    \item $\dfrac{\Gamma_1 \vdash A;\ \Gamma_2 \vdash \neg A}{\Gamma_1, \Gamma_2 \vdash}$ (\textit{сведение к противоречию});
    \item $\dfrac{\Gamma, \neg A \vdash}{\Gamma \vdash A}$ (\textit{удаление $\neg$});
    \item $\dfrac{\Gamma \vdash}{\Gamma \vdash A}$ (\textit{утончение});
    \item $\dfrac{\Gamma \vdash A}{\Gamma, B \vdash A}$ (\textit{расширение});
    \item $\dfrac{\Gamma_1, A, B, \Gamma_2 \vdash C}{\Gamma_1, B, A, \Gamma_2 \vdash C}$ (\textit{перестановка});
    \item $\dfrac{\Gamma, A, A \vdash C}{\Gamma, A \vdash C}$ (\textit{сокращение}).
\end{enumerate}

\subsubsection{Исчисление Клини}
\textbf{Примитивные связки}: $\neg$, $\land$, $\lor$, $\supset$.

\textbf{Схемы аксиом}:
\begin{enumerate}[label=(А\arabic*)]
    \item $A \supset (B \supset A)$;
    \item $(A \supset B) \supset \Big(\big(A \supset (B \supset C)\big) \supset (A \supset C)\Big)$;
    \item $A \land B \supset A$;
    \item $A \land B \supset B$;
    \item $(A \supset C) \supset \big((A \supset B) \supset (A \supset B \land C)\big)$;
    \item $A \supset (A \lor B)$;
    \item $B \supset (A \lor B)$;
    \item $(A \supset C) \supset \Big((B \supset C) \supset \big((A \lor B) \supset C\big)\Big)$;
    \item $(A \supset \neg B) \supset (B \supset \neg A)$;
    \item $\neg\neg A \supset A$.
\end{enumerate}

\textbf{Правило вывода}: MP.