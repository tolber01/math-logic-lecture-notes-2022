\subsection{Алгебра высказываний}
\subsubsection{Высказывания}
\textit{Высказывание}~---~утверждение, о котором можно определённо сказать,
истинно оно или ложно.

Высказывание может принимать только два истинностных значения: истина или ложь.

Высказывания делятся на \textit{простые} (\textit{элементарные}) и сложные (\textit{составные}). Простые высказывания представляют собой одно утверждение, сложные составлены из простых с помощью \textit{операций над высказываниями}.

Назначение логики высказываний~---~определение истинностных значений сложных высказываний только на основе их структуры, т.е. безотносительно смысла высказывания.

\subsubsection{Пропозициональные связки}
Сложные высказывания строятся как истинностно-функциональные комбинации простых высказываний.

Простые высказывания будем обозначать строчными латинскими
буквами: $a, b, c, \dots, a_1, a_2, \dots$, сложные~---~прописными латинскими буквами: $A, B, C, \dots, A_1, A_2, \dots$.

\textbf{Операции над высказываниями}:
\begin{enumerate}
    \item Отрицание.
    \begin{definition*}
        \textit{Отрицанием} высказывания $A$ называется высказывание $\neg A$, ложное тогда и только тогда, когда $A$ истинно.
    \end{definition*}
    Истинностные значения высказываний удобно записывать в таблицы~---~\textit{таблицы истинности} (\textit{истинностные таблицы}). \\
    Таблица истинности для отрицания:
    \begin{table}[h]
        \centering
        \begin{tabular}{| c | c | l}
            \cline{1-2} \HC $A$ & \HC $\neg A$ &  \\
            \cline{1-2}       И &            Л & И~---~истина, Л~---~ложь \\
            \cline{1-2}       Л &            И & \\
            \cline{1-2}
        \end{tabular}
    \end{table}

    \item Конъюнкция.
    \begin{definition*}
        \textit{Конъюнкцией} высказываний $A$ и $B$ называется высказывание $A \land B$, истинное тогда и только тогда, когда $A$ и $B$ истинны. 
    \end{definition*}
    Таблица истинности для конъюнкции:
    \begin{table}[h]
        \centering
        \begin{tabular}{| c | c | c |}
            \hline \HR $A$ & $B$ & $A \land B$ \\
            \hline       И &   И & И \\
            \hline       Л &   И & Л \\
            \hline       И &   Л & Л \\
            \hline       Л &   Л & Л \\
            \hline
        \end{tabular}
    \end{table}

    \item Дизъюнкция.
    \begin{definition*}
        \textit{Дизъюнкцией} высказываний $A$ и $B$ называется высказывание $A \lor B$, ложное тогда и только тогда, когда $A$ и $B$ ложны. 
    \end{definition*}
    Таблица истинности для дизъюнкции:
    \begin{table}[h]
        \centering
        \begin{tabular}{| c | c | c |}
            \hline \HR $A$ & $B$ & $A \lor B$ \\
            \hline       И &   И & И \\
            \hline       Л &   И & И \\
            \hline       И &   Л & И \\
            \hline       Л &   Л & Л \\
            \hline
        \end{tabular}
    \end{table}

    \item Импликация.
    \begin{definition*}
        \textit{Импликацией} высказываний $A$ и $B$ называется высказывание $A \supset B$, ложное тогда и только тогда, когда $A$ истинно, а $B$ ложно.
    \end{definition*}
    Таблица истинности для импликации:
    \[
        \begin{array}{@{} l @{}}
            \begin{array}{ | c | c | c | }
              \hline \HC \makebox[1em]{$A$} & \HC \makebox[1em]{$B$} & \HC \makebox[2.5em]{$A \supset B$} \\
              \hline               \text{И} &               \text{И} & \text{И} \\
              \hline               \text{И} &               \text{Л} & \text{Л}
            \end{array} \\
            \left.\kern-\nulldelimiterspace
            \begin{array}{ | c | c | c | }
              \hline \makebox[1em]{Л} & \makebox[1em]{И} & \makebox[2.5em]{\text{И}} \\
              \hline         \text{Л} &         \text{Л} & \text{И} \\
              \hline
            \end{array}
            \right\} \mbox{Принцип материальной импликации}
        \end{array}
    \]
    \begin{definition*}
        Высказывание $A$ называется \textit{антецедентом} (\textit{посылкой}), $B$~---~\textit{консеквентом} (\textit{следствием}) импликации.
    \end{definition*}

    \item Эквиваленция.
    \begin{definition*}
        \textit{Эквиваленцией} высказываний $A$ и $B$ называется высказывание $A \equiv B$, истинное тогда и только тогда, когда $A$ и $B$ принимают одинаковые истинностные значения.
    \end{definition*}
    Таблица истинности для эквиваленции:
    \begin{table}[h]
        \centering
        \begin{tabular}{| c | c | c |}
            \hline \HR $A$ & $B$ & $A \equiv B$ \\
            \hline       И &   И & И \\
            \hline       Л &   И & Л \\
            \hline       И &   Л & Л \\
            \hline       Л &   Л & И \\
            \hline
        \end{tabular}
    \end{table}
\end{enumerate}
\textit{Пропозициональными связками} называются знаки операций $\neg$, $\land$, $\lor$, $\supset$, $\equiv$.

\subsubsection{Формулы алгебры высказываний}\label{par:formulas_boolean}
Высказывания и операции над ними образуют \textit{алгебру высказываний}. \\
Для записи формул этой алгебры используем алфавит, состоящий из:
\begin{enumerate}
    \item строчных латинских букв $a, b, c, \dots, a_1, a_2, \dots$~---~\textit{пропозициональных букв}; 
    \item пропозициональных связок;
    \item специальных символов $($, $)$.
\end{enumerate}
\begin{definition*}
    \textit{Формулой алгебры высказываний} (\textit{пропозициональной формой}) называется слово в алфавите алгебры высказываний, построенное по правилам:
    \begin{enumerate}
        \item Любая пропозициональная буква есть формула.
        \item Если $A$, $B$ есть формулы, то слова $(\neg A)$, $(A \land B)$, $(A \lor B)$, $(A \supset B)$, $(A \equiv B)$ также являются формулами.
        \item Слово является формулой в том и только том случае, когда оно получено по правилам 1 и 2.
    \end{enumerate}
\end{definition*}
\begin{definition*}
    \textit{Подформулой} формулы называется её часть, сама являющаяся формулой.
\end{definition*}
\textbf{Правила удаления лишних скобок}:
\begin{enumerate}
    \item Внешние скобки можно опускать.
    \item Если формула содержит вхождения только одной бинарной связки $\land$, $\lor$, $\supset$ или $\equiv$, то для каждого вхождения можно опускать внешние скобки у подформулы слева.
    \item Введём приоритет связок (по возрастанию): $\equiv$, $\supset$, $\lor$, $\land$, $\neg$. Можно опускать пары скобок, без которых возможно восстановление исходной формулы по следующим правилам. Каждое вхождение связки $\neg$ относится к наименьшей следующей за ним подформуле. После расстановки скобок, относящихся к $\neg$ каждое вхождение символа $\land$ связывает наименьшие окружающие его подформулы. После расстановки скобок, относящихся к $\land$, каждое вхождение $\lor$ относится к наименьшим подформулам слева и справа от него. Далее подобным образом расставляются скобки, относящиеся к символам $\supset$ и $\equiv$. При применении этого правила к одинаковым связкам движение по формуле происходит слева направо.
\end{enumerate}
Формулы представляют собой формализованную математическую запись реальных высказываний. Поэтому для обозначения формул будем использовать прописные латинские буквы.

Каждому распределению истинностных значений пропозициональных букв, входящих в формулу, соответствует некоторое истинностное значение этой формулы, полученное по таблицам истинности пропозициональных связок. Таким образом, любая пропозициональная форма (слово, последовательность символов) определяет некоторую истинностную функцию (математическую функцию, функцию алгебры логики). Эта функция может быть графически представлена истинностной таблицей формулы.

Пример таблицы для формулы $\neg(A \land \neg B) \supset C$:
\begin{table}[h]
    \centering
    \begin{tabular}{| c | c | c | c | c | c | c |}
        \hline \HR $A$ & $B$ & $C$ & $\neg B$ & $A \land \neg B$ & $\neg(A \land \neg B)$ & $\neg(A \land \neg B) \supset C$ \\
        \hline       И &   И &   И &        Л &                Л &                      И & И \\
        \hline       И &   И &   Л &        Л &                Л &                      И & Л \\
        \hline       И &   Л &   И &        И &                И &                      Л & И \\
        \hline       И &   Л &   Л &        И &                И &                      Л & И \\
        \hline       Л &   И &   И &        Л &                Л &                      И & И \\
        \hline       Л &   И &   Л &        Л &                Л &                      И & Л \\
        \hline       Л &   Л &   И &        И &                Л &                      И & И \\
        \hline       Л &   Л &   Л &        И &                Л &                      И & Л \\
        \hline
    \end{tabular}
\end{table}

\subsubsection{Тавтологии}
Далее будем отождествлять форму и соответствующую ей истинностную функцию (не забывая при этом в чём их различие).
\begin{definition*}
    Формула называется \textit{тождественно истинной} (\textit{тавтологией}), если она истинна при любых наборах истинностных значений входящих в неё букв.
\end{definition*}
\begin{definition*}
    Формула называется \textit{тождественно ложной} (\textit{противоречием}), если она ложна при любых наборах истинностных значений входящих в неё букв.
\end{definition*}
\begin{definition*}
    Формула называется \textit{выполнимой} (\textit{опровержимой}), если она истинна (ложна) при некотором наборе истинностных значений входящих в неё букв.
\end{definition*}
Очевидно следующее утверждение.
\begin{lemma}
    Формула $A$ является тавтологией тогда и только тогда, когда $\neg A$ является противоречием.
\end{lemma}
Следующие важные теоремы служат основаниями для фундаментальных правил логического вывода.

\begin{theorem}\label{th:modus_ponens}
    Если $A$, $A \supset B$~---~тавтологии, то $B$~---~также тавтология.
\end{theorem}
\begin{proof}
    От противного. Предположим, что $B$ не является тавтологией. Тогда существует набор истинностных значений входящих в $B$ букв, который реализует ложность $B$. В силу того, что $A$~---~тавтология, на указанном наборе $A$ будет истинно. С другой стороны, импликация $A \supset B$ ложна в связи с истинностью $A$ и ложностью $B$ на указанному наборе, что вступает в противоречие с тем, что $A \supset B$~---~тавтология. \\
    Значит $B$ является тавтологией.
\end{proof}

Эта теорема обосновывает правило вывода по индукции modus ponens.

\begin{theorem}\label{th:tautology_substitution}
    Если $A$~---~тавтология, содержащая пропозициональные буквы $a_1, a_2, \dots, a_n$, формула $B$ получена подстановкой в $A$ формул $A_1, A_2, \dots, A_n$ вместо всех вхождений букв $a_1, a_2, \dots, a_n$ соответственно. Тогда $B$ также является тавтологией.
\end{theorem}
\begin{proof}
    От противного. Пусть $B$ не является тавтологией, тогда существует набор истинностных значений входящих в $B$ букв, реализующий ложность этой формулы. Пусть этот набор доставляет формулам $A_1, A_2, \dots, A_n$ истинностные значения $\alpha_1, \alpha_2, \dots, \alpha_n$ соответственно. Присвоим буквам $a_1, a_2, \dots, a_n$ формулы $A$ истинностные значения $\alpha_1, \alpha_2, \dots, \alpha_n$ соответственно. Ясно, что полученное ранее истинностное значение $B$ совпадает с истинностным значением $A$, полученным в предыдущей подстановке. Такое совпадение порождает противоречие, ибо $B$, как показано ранее, ложно, а $A$~---~тавтология по условию. \\
    Значит $B$ является тавтологией.
\end{proof}

Теорема \ref{th:tautology_substitution} утверждает, что подстановка в тавтологию вместо всех вхождений букв (причём, не обязательно всех букв) произвольных формул даёт тавтологию. Таким образом, она обосновывает правило подстановки, используемое неявно в рассматриваемых далее исчислениях.