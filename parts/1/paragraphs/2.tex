\subsection{Формальные аксиоматические теории}
\subsubsection{Определение формальной теории}
Метод формальных теорий~---~другой, более мощный метод решения задачи логических исчислений. Но вместе с тем, это очень трудный метод.

Формальная аксиоматическая теория определена, если:
\begin{enumerate}
    \item Задан \textit{алфавит} теории (алфавит~---~не более чем счётное множество символов).
    \item Задано подмножество слов в алфавите теории, которые считаются \textit{формулами} теории.
    \item Выделено некоторое подмножество формул~---~\textit{аксиом} теории.
    \item Задано конечное множество отношений между формулами теории, которые называются \textit{правилами вывода}.
\end{enumerate}

Введённые компоненты теории удовлетворяют следующим условиям:
\begin{enumerate}
    \item Можно эффективно определить, является ли данная формула аксиомой теории или нет. Именно такие теории будем называть \textit{аксиоматическими} (\textit{аксиоматизируемыми}).
    \item Правила вывода заданы эффективно. Это означает, что для каждого правила $R_i$ существует такое число $j > 0$, что для любого набора $j$ формул $A_1, \dots, A_j$ и для любой формулы $A$ можно эффективно определить, находятся ли эти формулы в отношении $R_i$ с формулой $A$: $\langle A_1; \dots; A_j; A\rangle \in R_i$. Если находятся, то говорят, что формула $A$ является непосредственным следствием формул $A_1, \dots, A_j$ по правилу вывода $R_i$:
    \[
        \frac{A_1, \dots, A_j}{A}.
    \]
\end{enumerate}

\subsubsection{Доказательства и теоремы}
\begin{definition*}
    \textit{Доказательством} (\textit{выводом}) в теории называется такая последовательность формул $A_1, \dots, A_m$, что для любого $i  > 0$ $A_i$~---~либо аксиома, либо непосредственное следствие каких-либо предыдущих формул по одному из правил вывода.
\end{definition*}
\begin{definition*}
    Формула называется \textit{теоремой} теории, если существует вывод, в котором эта формула последняя. Такой вывод называется \textit{доказательством} (\textit{выводом}) теоремы. 
\end{definition*}
\begin{definition*}
    Теория называется \textit{разрешимой}, если для любой формулы существует эффективный алгоритм определения, является ли она теоремой теории или нет.
\end{definition*}
В разрешимой теории доказательство можно автоматизировать (механизировать). В неразрешимой теории поиск доказательств~---~творческий процесс, посильный только человеку.
\begin{definition*}
    Формула $A$ называется \textit{следствием} в теории множества формул $\Gamma$, если существует последовательность формул $A_1, \dots, A_m$, в которой $A_m$ есть $A$, а для каждого $i > 0$ $A_i$~---~либо аксиома, либо непосредственное следствие каких-либо предыдущих формул по одному из правил вывода, либо формула из $\Gamma$. Такой вывод называется \textit{доказательством} (\textit{выводом}) формулы $A$ из множества формул $\Gamma$. Формулы из множества $\Gamma$ называются \textit{гипотезами} (\textit{посылками}).
\end{definition*}
Обозначается выводимость $\Gamma \vdash A$ или $A_1, \dots, A_s \vdash A$. Если $\Gamma = \varnothing$, то $\Gamma \vdash A$ равносильно тому, что $A$~---~теорема, поэтому тот факт, что $A$ является теоремой, записывают $\vdash A$.

\textbf{Свойства выводимости из посылок}:
\begin{enumerate}
    \item Если $\Gamma \subseteq \Delta$ и $\Gamma \vdash A$, то $\Delta \vdash A$ (в множество гипотез можно добавлять любые формулы).
    \item $\Gamma \vdash A$ тогда и только тогда, когда в $\Gamma$ имеется такое конечное подмножество $\Delta$, что $\Delta \vdash A$ (некоторые формулы можно удалять из множества гипотез без потери выводимости).
    \item Если $\Delta \vdash A$ и $\Gamma \vdash B$ для каждой формулы $B \in \Delta$, то $\Gamma \vdash A$.
\end{enumerate}